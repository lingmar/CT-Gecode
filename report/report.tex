\documentclass[a4paper,11pt]{article}
\usepackage{fullpage}
\usepackage{astra}
%\usepackage{etoolbox}\AtBeginEnvironment{algorithmic}{\small‌​}
%\usepackage{algorithm2e}
%\usepackage{algorithmicx}
%\input{macros}

\usepackage{amsthm}

\newtheorem{definition}{Definition}

% Silly but saves space
\newcommand{\T}[1]{\texttt{#1}}

\newcommand{\Timeout}{600.00} % CPU seconds
\newcommand{\Todo}[1]{{\color{blue}Todo: #1}}
\newcommand{\Secref}[1]{Section~\ref{#1}}
\newcommand{\Chapref}[1]{Section~\ref{#1}}
\newcommand{\Algoref}[1]{Algorithm~\ref{#1}}
\newcommand{\Table}{\Constraint{Table}~}
\newcommand{\Extensional}{\Constraint{Extensional}~}

\newcommand{\Method}[2]{\textbf{method~}\mathrm{{#1}}({#2})}
\newcommand{\MethodReturn}[3]{\textbf{method~}\mathrm{{#1}}({#2})\textbf{\ : \ {#3}}}
\newcommand{\Class}{\textbf{Class~}}
\newcommand{\Constructor}{\textbf{constructor~}}

\newcommand{\Dom}[1]{\text{dom}({#1})}

% SparseBitSet
\newcommand{\Words}{\texttt{words}}
\newcommand{\Index}{\texttt{index}}
\newcommand{\Mask}{\texttt{mask}}
\newcommand{\Limit}{\texttt{limit}}
\newcommand{\SparseBitSet}{\texttt{SparseBitSet}}
\newcommand{\Offset}{\texttt{offset}}

% CT Propagator
\newcommand{\Scp}{\texttt{scp}}
\newcommand{\CurrTable}{\texttt{currTable}}
\newcommand{\Sval}{\texttt{S^{val}}}
\newcommand{\Ssup}{\texttt{S^{sup}}}
\newcommand{\LastSizes}{\texttt{lastSize}}
\newcommand{\Supports}{\texttt{supports}}
\newcommand{\Residues}{\texttt{residues}}

% Pseduo code
\newcommand{\ForEach}[1]{\textbf{foreach } {#1} \textbf{ do }}
\newcommand{\ForEachTo}[3]{\textbf{foreach } {#1} \textbf{ from } {#2} 
  \textbf{ to } {#3} \textbf{ do }}
\newcommand{\ForEachDownTo}[3]{\textbf{foreach } {#1} \textbf{ from } {#2} 
  \textbf{ downto } {#3} \textbf{ do }}
\newcommand{\Break}{\textbf{break~}}
\newcommand{\While}[1]{\textbf{while~} {#1} \textbf{~do~}}

\renewcommand{\algorithmicforall}{\textbf{Method}}
\renewcommand{\algorithmicdo}{}
\renewcommand{\algorithmicwhile}{\textbf{foreach}}

\newcommand{\Func}[2]{\FORALL{#1(#2)}}
\newcommand{\FuncRet}[3]{\FORALL{#1(#2) \ : \ \textbf{#3}}}
\newcommand{\Endfunc}{\ENDFOR}
\newcommand{\To}{~\bf{to}~}
\newcommand{\Downto}{~{\bf{downto}}~}
\newcommand{\For}[3]{\FOR{${#1} \leftarrow {#2} \To {#3}$ \textbf{do}}}
\newcommand{\ForDown}[3]{\FOR{${#1} \leftarrow {#2} \Downto {#3}$ \textbf{do}}}
\newcommand{\FOREACH}[1]{\WHILE{{#1} \textbf{do}}}
\newcommand{\ENDFOREACH}{\ENDWHILE}

\newcommand{\function}[1]{\mathrm{#1}}
\newcommand{\localvar}[1]{\mathit{#1}}

\newlength\myindent
\setlength\myindent{2em}
\newcommand\bindent{%
  \begingroup
  \setlength{\itemindent}{\myindent}
  \addtolength{\algorithmicindent}{\myindent}
}
\newcommand\eindent{\endgroup}

\newcommand{\INDSTATE}[1][1]{\STATE\hspace{#1\algorithmicindent}}
\newcommand{\INDRETURN}[1][1]{\STATE\hspace{#1\algorithmicindent}\textbf{return~}}
\newcommand{\INDIF}[2][1]{\STATE\hspace{#1\algorithmicindent}
  \textbf{if~}{#2}\textbf{~then}}
\newcommand{\INDELSE}[1][1]{\STATE\hspace{#1\algorithmicindent}\textbf{else~}}
\newcommand{\INDELSEIF}[2][1]{\STATE\hspace{#1\algorithmicindent}
  \textbf{else if~}{#2}\textbf{~then}}

\newcommand{\CTpaper}[0]{DBLP:conf/cp/DemeulenaereHLP16}

\title{\textbf{Implementation and Evaluation of a\\
    Compact Table Propagator in Gecode
  }
}

\author{Linnea Ingmar} % replace by your name(s)

%\date{Month Day, Year}
\date{\today}

\begin{document}

\maketitle

\tableofcontents

\newpage

\section{Introduction}
\label{intro}

In Constraint Programming (CP), every constraint is associated with a propagator
algorithm. The propagator algorithm filters out impossible values for the variables
related to the constraint. For the \Table constraint, several propagator
algorithms are known. In 2016, a new propagator algorithm for the \Table
constraint was published~\cite{\CTpaper}, called Compact Table (CT).
Preliminary results indicate that CT outperforms the previously known algorithms.
There has been no attempt to implement CT in the constraint solver Gecode~\cite{Gecode}, and consequently its performance in Gecode is unknown.

\subsection{Goal}
\label{intro:goal}
The goal of this thesis is to implement a CT progatator
algorithm for the \Table constraint in Gecode,
and to evaluate its performance with respect to the existing propagators.

\subsection{Contributions}
\label{intro:contributions}

\Todo{State the contributions, perhaps as a bulleted list, referring to the different
parts of the paper, as opposed to giving a traditional outline. (As suggested
by Olle Gallmo.)}

This thesis contributes with the following:

\begin{itemize}
  \item The relevant preliminaries have been covered in \Chapref{bg}.
  \item The algorithms presented in~\cite{DBLP:conf/cp/DemeulenaereHLP16} have been modified to suit the
    target constraint solver Gecode, and are presented and explained in 
    \Chapref{algorithms}.
  \item The CT algorithm has been implemented in Gecode, see \Chapref{implementation}.
  \item The performance of the CT algorithm has been evaluated, see \Chapref{evaluation}.
  \item ...
\end{itemize}

\section{Background}
\label{bg}

% Definiera alla begrepp som används senare

This chapter provides a background that is relevant for the
following chapters. It is divided into three parts: \Secref{bg:cp}
introduces Constraint Programming. \Secref{bg:gecode} gives an overview
of Gecode, a constraint solver. Finally, \Secref{bg:table} introduces
the \Table constraint.

\subsection{Constraint Programming}
\label{bg:cp}
This section introduces the concept of Constraint Programming (CP).

CP is a programming paradigm that is used for solving
combinatorial problems. A problem is
modelled as a set of \emph{constraints} and a
set of \emph{variables} with possible values. The possible values of 
a variable is called the \emph{domain} of the variable.
All the variables are to be assigned a value
from their domains, so that all the constraints of the problem
are satisfied.
%  There need not
% exist a solution to a given problem, and a solution need not
% be unique; zero or more assignments of values to the variables may 
% satisfy the constraints.
One small example is the set of variables~$\Set{x,y}$ with
domain~$\Set{1,2,3}$,
together with the constraint~$x<y$. This problem has two solutions, 
namely~$(x,y) = (1,2), (x,y) = (1,3)$ and~$(x,y) = (2,3)$.
If, on the other hand, the domains of~$x$ and~$y$ would have been~$\Set{2,3}$
and~$\Set{1,2}$, respectively, then there would exist no solution since none of
the possible assignments of~$x$ and~$y$ would have satisfied the constraint~$x<y$.

Sometimes the solution should not only satisfy the set of constraints for the
problem, but should also maximise or minimise some given function.
Take the same small example as above, with
the domains of~$x,y$ being~$\Set{1,2,3}$, and suppose that the solution
should maximise the sum of~$x$ and~$y$. Then the solution is~$(x,y) = (2,3)$.

A constraint solver is a software that solves constraint problems.
The solving of a problem consists of generating a search tree by branching
on possible values for the variables. At each node in the search tree,
the solver removes impossible values from the domains of the variables.
This filtering process is called \emph{propagation}. Each constraint is
associated with at least one propagator algorithm, whose task is to detect
values that would violate the constraint if the variables were to be assigned
any of those values, and remove those values from the domain of the variable.

After this intuitive description we are ready to define some concepts.

% Intuition
% Example
% Definitions
% - CSP

% Propagation
% - Constistency levels (value-, bounds-, domain-)
% 

\paragraph{Propagation}
% From chap14

\begin{definition}
  A domain~$D$ is a complete mapping from a fixed (countable) set of 
  variables~$V$ to finite sets of integers. A domain~$D$ is failed,
  if $D(x) = \emptyset$ for some $x \in V$. A variable~$x \in V$
  is fixed by a domain D, if $|D(x)| = 1$. The intersection of domains~$D1$ and~$D2$,
  denoted~$D1 \sqcap D2$, is defined by the domain
  $D(x) = D1(x) \Union D2(x)$ for all $x \in V$.
  A domain~$D1$ is stronger than a domain~$D2$, written $D1 \sqsubseteq D2$,
  if~$D1(x) \subset D2(x)$ for all $x \in V$.
\end{definition}

A propagator~$p$ is a function that maps domains to domains.

\subsection{Gecode}
\label{bg:gecode}
Gecode~\cite{Gecode} is a popular constraint programming solver written in C++.

% definiera de delar av Gecodes API som dyker upp senare, såsom propagate(), status messages
% använda inbyggda klasen BitSets?

\subsection{The \Table Constraint}
\label{bg:table}
The \Table constraint, called \Extensional in Gecode,
explicitly expresses the possible combinations of values for the variables as a
sequence of $n$-tuples.

\subsection{Compact-Table Propagator}
\label{bg:ct}
% Beskriv huvudidéerna
% Komplexitet? Kolla artikeln om negativa table-villkor
% O(r*d*t) per table constraint along a branch in the search tree (artikeln om bakgrund)

\subsection{Sparse Bit-Set}
\label{bg:sbc}
% Beskriv idén

\section{Algorithms}
\label{algorithms}

% Section 3 bör beskriva din design i detalj men samtidigt inte på C++-nivå. Jag gillar att se sjok av pseudokod inbäddade i text som förklarar pseudokoden. Man kan skriva text mellan sjoken och/eller i caption till algorithm-omgivningen. Något som jag också gillar är stepwise refinement, dvs. att först visa en enkel men korrekt version, och sedan en eller flera mer sofistikerade, optimerade versioner. Den pseudokod som du har skrivit passar bra i Section 3, men bryt gärna upp åtminstone Class CT-Propagator i flera stycken algorithm-omgivningar.

This chapter presents the algorithms that are used in the implementation of the
CT propagator in \Chapref{implementation}.

\subsection{Sparse Bit-Set}
This section describes the class~$\SparseBitSet$ which is the main datastructure
in the CT algorithm for maintaining the supports.

\begin{algorithm}[h]
  \begin{algorithmic}[1]  % comment [1] away to drop the line numbers
        \STATE $\Class$ SparseBitSet
    \item[]
      \STATE $\Words$: array of long \COMMENT{$\Words.\function{length}() = p$} \label{line:sbsfield:start}
      \STATE $\Index$: array of int \COMMENT{$\Index.\function{length}() = p$}
      \STATE $\Limit$: int 
      \STATE $\Mask$: array of long \COMMENT{$\Mask.\function{length}() = p$} \label{line:sbsfield:end}

    \item[]
    \Func{initSparseBitSet}{$\localvar{nbits}$: int} \label{line:initsbs:start}
      \STATE $\localvar{p} \leftarrow \Ceiling{\frac{\localvar{nbits}}{64}}$
      \STATE $\Words \leftarrow \text{~array of long of length~} \localvar{p} \text{, first~} 
      \localvar{nbits} \text{~set to 1}$
      \STATE $\Mask \leftarrow \text{~array of long of length} \localvar{p} \text{, all bits set to~}0$
      \STATE $\Index \leftarrow [0, \ldots, \localvar{p} - 1]$
      \STATE $\Limit \leftarrow \localvar{p-1}$ \label{line:initsbs:end}
      \Endfunc

    \item[]
      \FuncRet{isEmpty}{}{Boolean} \label{line:isEmpty:1}
      \RETURN{$\Limit = -1$} \label{line:isEmpty:2}
      \Endfunc
    \item[]
      \Func{clearMask}{} \label{line:clearMask:1}
      \For{i}{0}{\Limit} \label{line:clearMask:2}
        \STATE $\localvar{offset} \leftarrow \Index[i]$ \label{line:clearMask:3}
        \STATE $\texttt{mask}[\localvar{offset}] \leftarrow 0^{64}$ \label{line:clearMask:4}
      \ENDFOR
      \Endfunc

    \item[]
      \Func{reverseMask}{} \label{line:reverse:1}
      \For{i}{0}{\Limit} \label{line:reverse:2}
      \STATE $\Offset \leftarrow \Index[i]$ \label{line:reverse:3}
      \STATE $\texttt{mask}[\Offset] \leftarrow ~\texttt{mask}[\Offset]$ \COMMENT{birwise NOT} \label{line:reverse:4}
      \ENDFOR
      \Endfunc



    % \item[] 
    %   \Func{reverseMask}{}  \COMMENT{Not currently used in CT algorithm}
    %   \STATE $\ForEachTo{i}{0}{\Limit}$
    %   \STATE $\localvar{offset} \leftarrow \Index[i]$
    %   \STATE $\texttt{mask}[\localvar{offset}] \leftarrow 
    %   {\raise.17ex\hbox{$\scriptstyle\mathtt{\sim}$}}
    %   \texttt{mask}[\localvar{offset}]$ \COMMENT{bitwise NOT}
      %\Endfunc

    \item[]
      \Func{addToMask}{$\localvar{m}$: array of long} \label{line:addToMask:1}
      \For{i}{0}{\Limit} \label{line:addToMask:2}
      \STATE $\localvar{offset} \leftarrow \Index[i]$ \label{line:addToMask:3}
      \STATE $\texttt{mask}[\localvar{offset}] \leftarrow \texttt{mask}[\localvar{offset}] \ | \ 
      \localvar{m}[\localvar{offset}]$ \COMMENT{bitwise OR} \label{line:addToMask:4}
      \ENDFOR
      \Endfunc

    \item[]
      \Func{intersectWithMask}{} \label{line:intersect:1}
      %\FOR{$i \leftarrow \Limit \Downto 0$}
      \ForDown{i}{\Limit}{0} \label{line:intersect:2}
      \STATE $\localvar{offset} \leftarrow \Index[i]$ \label{line:intersect:3}
      \STATE $w \leftarrow \Words[\localvar{offset}] \ \& \ \Mask[\localvar{offset}]$ \COMMENT{bitwise AND} \label{line:intersect:4}
      \IF{$w \neq \Words[\localvar{offset}]$} \label{line:intersect:5}
      \STATE $\Words[\localvar{offset}] \leftarrow w$ \label{line:intersect:6}
      \IF{$w = 0^{64}$} \label{line:intersect:7}
      \STATE $\Index[i] \leftarrow \Index[\Limit]$ \label{line:intersect:8}
      \STATE $\Index[\Limit] \leftarrow \localvar{offset}$ \label{line:intersect:8.5}
      \STATE $\Limit \leftarrow \Limit - 1$ \label{line:intersect:9}
      \ENDIF
      \ENDIF
      \ENDFOR
      \Endfunc

    \item[]
      \FuncRet{intersectIndex}{$\localvar{m}$: array of long}{int} \label{line:interIdx:1}
      \For{i}{0}{\Limit} \label{line:interIdx:2}
      \STATE $\Offset \leftarrow \Index[i]$ \label{line:interIdx:4}
      \IF{$\Words[\Offset] \ \& \ m[\Offset] \neq 0^{64}$} \label{line:interIdx:5}
      \RETURN{$\Offset$} \label{line:interIdx:6}
      \ENDIF
      \ENDFOR
      \RETURN{$-1$} \label{line:interIdx:7}
      \Endfunc


    \end{algorithmic}
  \caption{Pseudo code for the class SparseBitSet.}
  \label{algo:sparse}
\end{algorithm}

\subsection{Compact-Table (CT) Algorithm}
This section describes the CT algorithm.

Compact-Tablce (CT) is a domain consistent propagation algorithm 
for a table constraint~$c$. It dynamically
maintains a set of valid supports regarding the current domain of each variable.
This description is mainly taken from~\cite{CTpaper}.

% Beskriv övergripande?

% CT is based on bitwise operations -- among the important datastructures we have
% a~$\SparseBitSet$ object which maintains the valid supports, and also a bitset
% matrix that

% In this ordered set, each tuple is indexed
% by the order it appears in the table, and the~$i$th element is~$1$ if and only if
% the~$i$th tuple is a valid support, else the~$i$th element is~$0$.

\subsubsection{Fields}

\Todo{Add examples with figures for describing the fields.}

The fields of the class \texttt{CT-Propagator} are as follows:

\begin{itemize}
  \item $\Scp$ represents the scope of the constraint~$c$.
  \item $\CurrTable$ represents the current table,
    that is, the valid supports for~$c$. If the initial table of~$c$ is
    $\Tuple{\tau_0, \tau_1, ..., \tau_{p-1}}$, then~$\CurrTable$ is a 
    $\SparseBitSet$ object of initial size~$p$, such that value~$i$
    is contained (is set to~$1$) if and only if the~$i$th tuple is valid:
    
    \begin{equation}
      i \in \CurrTable \ \Leftrightarrow \ \forall x \in scp(c): \tau_i[x] \in \Dom{x}
    \end{equation}

  \item $\Supports$ represents the supports for each variable-value pair~$(x,a)$,
    where~$x \in scp(c) \land a \in \Dom{x}$.
    It is a static array of words~$\Supports[x,a]$, seen as bit-sets.
    The main idea is that the bit-set~$\Supports[x,a]$ is such that
    the bit at position~$i$ is set to~$1$ if and only if the 
    tuple~$\tau_i$ in the initial table of~$c$ is a support for~$(x,a)$:

    \begin{equation}
      \forall x \in scp(c): \forall a \in \Dom{x}:
      \Supports[x,a][i] = 1 \Leftrightarrow \tau_i[x] = a \ \land \
      \forall y \in scp(c): \tau_i[y] \in \Dom{y}
    \end{equation}

    One can optimise this definition slighty since it is unnecessary to keep track of
    invalid supports. Instead, only the tuples that are valid supports for~$c$
    need to be indexed in~$\Supports[x,a]$, potentially making the bit-set shorter
    if only a subset of the tuples in the initial table are valid supports. 
    If the initial table 
    is~$T = \Tuple{\tau_0, \tau_1, ..., \tau_{p-1}}$ and it turns out that
    $\tau_0,...,\tau_{i-1}$ are valid supports and~$\tau_{i}$ is invalid, then~$\tau_{i+1}$
    will correspond to index~$i$ instead of~$i+1$ in~$\Supports[x,a]$. So a better
    definition reads:

    \begin{equation}
      \forall x \in scp(c): \forall a \in \Dom{x}:
      \Supports[x,a][j] = 1 \Leftrightarrow \tau_i[x] = a \ \land \
      \forall y \in scp(c): \tau_i[y] \in \Dom{y}
    \end{equation}
    
    where~$j = |\Set{\tau_k \ | \ \tau_k \in T \ \land \ \tau_k \text{~is valid} \ \land \ k < i}|$.
    Seing~$\Supports$ as a matrix, we have that the column~$\Supports[*,*][i]$ encodes
    the~$i$th valid support for~$c$.

    $\Supports$ is computed once during the initialisation of CT and then
    remains unchanged.
    
  \item $\LastSizes$ is an array that contains the domain size of each
    variable~$x$ right before the previous invocation of CT on~$c$.
    % Not upon initialisation though

  \item $\Residues$ is an array such that for each variable-value pari~$(x,a)$,
    $\Residues[x,a]$ denotes the index of the word in~$\CurrTable$ where a support
    was found for~$(x,a)$ the last time it was sought for.

\end{itemize}

The initialisation of the fields of Class \texttt{CT-Propagator} is described in
\Algoref{algo:CT-initialise}. The method initialiseCT() takes two parameters:
\texttt{variables}, that are the variables associated to the constraint~$c$,
and~\texttt{tuples}, that is a list of tuples that define the initial table for~$c$.

\newcommand{\Lineref}[1]{Line~\ref{#1}}
\newcommand{\Linesref}[2]{Lines~\ref{#1}-\ref{#2}}

\Linesref{line:init:1}{line:init:2}~initialise~\Scp~and~\LastSizes.
The array~\LastSizes~is filled with 
the dummy value~$-1$ because we
want to have~$|\Dom{x}| \neq \LastSizes[x]$ the first time that the propagation
algorithm runs.

\Linesref{line:init:3}{line:init:4}~initialise local variables that will be 
used later.

\Lineref{line:init:5}~initialises~\Supports~as an array of bit-sets with
one bit-set for each variable-value pair and the size of each bit-set being
the number of tuples in the initial table. Each bit-set is assumed to be
initially filled with zeros.

\Linesref{line:init:6}{line:init:7} set the correct bits to~$1$ in~$\Supports$.
For each tuple~$t$, we check if~$t$ is a valid support for~$c$. Recall that~$t$ is
a valid support for~$c$ if and only if~$t[x] \in \Dom{x}$ for all~$x \in scp(c)$.
If~$t$ is a valid support,
all elements in~$\Supports$ corresponding to~$t$ are to be set to~$1$.
We keep a counter,~$no\_supports$, for the number of valid supports for~$c$.
This is used for indexing the tuples in~$\Supports$ in the way discussed above.
The method setElemsInColumn($i,t$) is assumed to set all values corresponding to
the values in~$t$ in the~$i$th column in~$\Supports$ to~$1$. 
\Todo{pseudocode for setElemsInColumn()}
\Todo{Perhaps break out lines 8-16 to an own algorithm-environment.}

\Todo{trimToWidth()}.

\Lineref{line:init:8} initialises~$\CurrTable$ as a~$\SparseBitSet$ object with
$no\_supports$ bits, initially with all bits set to~$1$ since~$i$ tuples are initially
valid supports for~$c$.

\Lineref{line:init:9} initialises~$\Residues$ as an array of ints filled with zeros.
Recall that~$\Residues[x,a]$ is the index of the word where the last found
support for~$(x,a)$ resides in~\CurrTable. Since we have not sought for any supports yet,
we let~$0$ be our start guess.
\Todo{Actually, we could initialise $\Residues[x,a]$ in the if-statement in line 14.}\\

 \begin{algorithm}%[h]
  \begin{algorithmic}[1]  % comment [1] away to drop the line numbers
      \Func{initialiseCT}{\texttt{variables}, \texttt{tuples}} %\COMMENT{Initialise the propagator}
      \STATE $\Scp \leftarrow \texttt{variables}$ \label{line:init:1}
      \STATE $\LastSizes \leftarrow \text{array of  length~} \Scp.\function{length}
      ~\text{filled with -1}$ \COMMENT{dummy value} \label{line:init:2}

      \STATE $\localvar{size} \leftarrow \function{sum}\Set{|\Dom{x}| : x \in \Scp}$ \label{line:init:3}
      \STATE $\localvar{ntuples} \leftarrow \texttt{tuples}.\function{size}()$ \COMMENT{Number of tuples}
      \STATE $\localvar{no\_supports} \leftarrow 0$ \label{line:init:4}
      \STATE $\Supports \leftarrow$ BitSets$(\localvar{size}, \localvar{ntuples})$ 
      \COMMENT{bit-set matrix with $\localvar{size}$ rows and $\localvar{ntuples}$ columns} \label{line:init:5}

      
      \FOREACH{$t \in \T{tuples}$} \label{line:init:6}
        \STATE $\localvar{supported} \leftarrow \T{true}$
        \FOREACH{$x \in \Scp$}
          \IF{$t[x] \notin \Dom{x}$}
            \STATE $\localvar{supported} \leftarrow \T{false}$
            \STATE $\textbf{break}$ \COMMENT{Exit loop}
          \ENDIF
        \ENDFOREACH
          \IF{$\localvar{supported}$} 
            \STATE $\localvar{no\_supports} \leftarrow \localvar{no\_supports} + 1$
            \STATE $\Supports.$setElemsInColumn($\localvar{no\_supports}$, $t$) 
          \ENDIF
      \ENDFOREACH \label{line:init:7}
      %\COMMENT{Mark tuple as supported}

      \STATE $\Supports.\text{trimToWidth}(\localvar{no\_supports})$ \COMMENT{Keep only the first $\localvar{no\_supports}$ bits for each row}
      \STATE $\CurrTable \leftarrow \SparseBitSet(\localvar{no\_supports})$ 
      \COMMENT{$\SparseBitSet$ with $\localvar{no\_supports}$ bits} \label{line:init:8}
      \STATE $\Residues \leftarrow \text{array of size~} \localvar{size}
      \text{~filled with 0}$ \label{line:init:9}
      \Endfunc
    \end{algorithmic}
  \caption{Pseudo code for initialising the CT-propagator.}
  \label{algo:initialise-CT}
  \end{algorithm}
 
  To keep down the size of~$\Supports$, one could perform simple bounds
  propagation in initialiseCT(). \Algoref{algo:initail-propagation}
  is inserted after line~$\ref{line:init:2}$ in initialiseCT(). It removes
  from the domain of each variable~$x$ all values that are either greater 
  than the largest element or smaller than the smallest element in the
  initial table.

\begin{algorithm}[h]
  \begin{algorithmic}[1]  % comment [1] away to drop the line numbers
    \FOREACH{$x \in \Scp$} %\COMMENT{Initial bounds propagation}
        \STATE $\Dom{x} \leftarrow \Dom{x} \setminus  
        \Set{a \in \Dom{x} : a > \texttt{tuples}.\function{max}()
          ~\text{or}~a < \texttt{tuples}.\function{min}()}$
      \ENDFOREACH
  \end{algorithmic}
  \caption{Simple initial propagation for keeping down the size of \Supports.}
  \label{algo:initial-propagation}
\end{algorithm}


 \begin{algorithm}[h]
  \begin{algorithmic}[1]  % comment [1] away to drop the line numbers
    \STATE $\Class$ CT-Propagator
    \item[]
      \STATE $\Scp$: array of variables
      \STATE $\CurrTable$: SparseBitSet \COMMENT{Current supported tuples}
      \STATE $\Supports$: array of BitSet   \COMMENT{$\Supports[x,a]$ is the bit-set of supports for $(x,a)$}
      \STATE $\LastSizes$: array of int \COMMENT{$\LastSizes[x]$ is the last size of the domain of $x$.c}
      \STATE $\Residues$: array of int  \COMMENT{$\Residues[x,a]$ is the last found support for $(x,a)$. 
      No residues yet!}
            
    \item[]
      \Func{updateTable}{}   
      \FOREACH{$x \in \Scp \text{~such that~} |\Dom{x}| \neq \LastSizes[x]$}
        \STATE $\LastSizes[x] \leftarrow |\Dom{x}|$
        \STATE $\CurrTable$.clearMask() 
        \FOREACH{$a \in \Dom{x}$} %\COMMENT{No incremental update yet!}
          \STATE $\CurrTable$.addToMask($\Supports[x,a]$)
          \STATE $\CurrTable$.intersectWithMask()
          \IF{$\CurrTable$.isEmpty()}
            \STATE $\Break$ \COMMENT{No valid tuples left}
          \ENDIF
        \ENDFOREACH
     \ENDFOREACH
     \Endfunc

    \item[]
      \Func{filterDomains}{}
      \STATE $\localvar{count\_unassigned} \leftarrow 0$
      \FOREACH{$x \in \Scp \text{~such that~} |\Dom{x}| > 1$}
            \FOREACH{$a \in \Dom{x}$}
                  \STATE $\texttt{index} \leftarrow \CurrTable$.intersectIndex($\Supports[x,a]$)
                  \IF{$\texttt{index~} \neq -1$}
                        \STATE \todo{save index in residues} \COMMENT{No residues yet!}
                  \ELSE
                        \STATE $\Dom{x} \leftarrow \Dom{x} \setminus \Set{a}$
                        \IF{$|\Dom{x}| > 1$}
                           \STATE $\localvar{count\_unassigned} \leftarrow \localvar{count\_unassigned} + 1$
                        \ENDIF
                  \ENDIF
             \ENDFOREACH
      \ENDFOREACH
      \IF{$\localvar{count\_unassigned} \leq 1$}
        \RETURN{$Subsumed$}
      \ELSE
        \RETURN{$Fixpoint$}
      \ENDIF
      \Endfunc

    \item[]
      \STATE $\Method{propagate}{}$
      \STATE updateTable()
      \IF{$\CurrTable$.isEmpty()}
        \RETURN{$Failed$}
      \ENDIF
      \RETURN{filterDomains()}

    \end{algorithmic}
  \caption{Pseudo code for CT propagator class.}
  \label{algo:CT}
\end{algorithm}

\subsubsection{Propagator Status Messages}
A propagtor signals a status message after propagation.
For CT there are three possible status messages;~\emph{Fail},
\emph{Subsumed} and~\emph{Fixpoint}.

\paragraph{Fail.}
A propagator must correctly signal failure
if it has decided that the constraint is unsatisfiable for the input store~$S$.
At the latest a propagator must be
able to decide whether or not a~$S$ is a solution store when all
variables have been assigned.
CT has two ways of detecting a failure: either when all bits in~$\CurrTable$
are set to zero -- meaning that none of the tuples are valid, or when the
size of the domain of a variable is zero.

\paragraph{Subsumption.}
A propagator is not allowed to signal subsumption if it could propagate further
at a later point. At the latest, a propagator must signal subsumption if all the variables
are assigned in~$S$ and a~$S$ is a solution store.
CT signals subsumption when at most one variable is not assigned,
since this is the pointe where no more propagation can be made.

\paragraph{Fixpoint.}
A propagator is not allowed to claim that it has computed a fixpoint if
it could still propagate.
CT always computes a fixpoint if it is not subsumed or wipes out the domain of a variable.
To understand this, consider two consequtive calls to propagate().
Let~$T$ be the set of valid tuples and~$x$ be the set of variables.
The first time propagate() is executed, a (possibly empty) subset~$T_r$ of~$T$
is invalidated (that is, the corresponding bits in $\CurrTable$ are set to $0$)
in updateTable().
Assuming $\CurrTable$ is not empty after the return of updateTable(), which would
cause a backtrack in the search, filterDomains() is called.
This method removes a (possibly empty) set of values~$V_i$ from the domain 
of each variable~$x_i$.
Each value $v_i \in V_i$ has the property that a subset~$T_{v_i}$ 
of the tuples in~$T_r$ are the last supports for~$(x_i,v_i)$. In other words,
$T \setminus T_{v_i}$ does not contain a support for~$(x_i,v_i)$.
So in the second call to updateTable(), no tuples are invalidated,
because none of the tuples in~$T \setminus T_r$ is a support for any variable-value
pair~$(x_i, v_i) \in \Set{x_i} \times V_i, i \in \Set{1...|x|}$.
Hence, the second call to propagate() does not give
any further propagation.

\section{Implementation}
\label{implementation}

% Section 4 blir nog mindre intressant än Section 3 och 5, men där kan du skriva om sånt som är specifikt för C++ och Gecode för att algoritmerna i Section 3 ska fungera, precis som du har börjat göra. Det är också en bra plats för detaljer som sopats under mattan i pseudokoden, t.ex. exakt hur du mappar (x,a) till rätt element i supports och residues, med hashtabell eller så.

This chapter describes an implementation of the CT propagator using the algorithms
presented in \Chapref{algorithms}. The implementation was made in the C++ programming
language in the Gecode library.

The bit-set matrix $\Supports$ is static and can be shared between all solution spaces.

The bit-set $\CurrTable$ changes dynamically during propagation and must therefore be copied for
every new space. Can save memory by only copying the non-zero words.

No need to save the $\T{tuples}$ as a field in the propagator class as all
the necessary information is encoded in $\CurrTable$ and $\Supports$.


% Can't modify the value of the variable while iterating over it
% when using an iterator for a view, the view cannot be modified (or, in C++ lingua: modifying the variable invalidates the iterator).

% The motivation to iterate over range sequences rather than individual values is efficiency:
% as there are typically less ranges than indvidual values, iteration over ranges can be more efficient.

% Sharing of domain and iterators (argument false in inter_v)

%http://www.gecode.org/doc/4.4.0/reference/classGecode_1_1Iter_1_1Values_1_1BitSet.html

% First perform bounds propagation

% Staging p. 324

% Region (memory allocation)

% Multimap for hashing rows?

\section{Evaluation}
\label{evaluation}
This chapter presents the evaluation of the implementation of the CT propagator
presented in \Chapref{implementation}. In \Secref{evaluation:setup},
the evaluation setup is described. In \Secref{evaluation:results} presents
the results of the evaluation. The results are discussed in \Secref{evaluation:discussion}.

\subsection{Evaluation Setup}
\label{evaluation:setup}
\subsection{Results}
\label{evaluation:results}
\subsection{Discussion}
\label{evaluation:discussion}

\section{Conclusions and Future Work}
\label{conclusions}

\bibliographystyle{abbrv}
\bibliography{astra}

\appendix
\section{Source Code}
\label{source code}

This appendix presents the source code for the implementation described in \Chapref{implementation}.

\end{document}

%%% Local Variables:
%%% mode: latex
%%% TeX-master: t
%%% End:

% OscaR source code:
% https://bitbucket.org/oscarlib/oscar/src/40e25aafba8f9b0ab06029449350a2a9d1614854/oscar-algo/src/main/scala/oscar/algo/reversible/ReversibleSparseBitSet.scala?at=dev&fileviewer=file-view-default
% https://bitbucket.org/oscarlib/oscar/src/40e25aafba8f9b0ab06029449350a2a9d1614854/oscar-cp/src/main/scala/oscar/cp/constraints/tables/TableCT.scala?at=dev&fileviewer=file-view-default3

% 