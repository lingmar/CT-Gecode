\documentclass[a4paper,11pt]{article}
\usepackage{fullpage}
\usepackage{astra}
%\usepackage{etoolbox}\AtBeginEnvironment{algorithmic}{\small‌​}
%\usepackage{algorithm2e}
%\usepackage{algorithmicx}
%\input{macros}

%\usepackage{amsthm}

\newtheorem{definition}{Definition}

% Silly but saves space
\newcommand{\T}[1]{\texttt{#1}}

\newcommand{\Timeout}{600.00} % CPU seconds
\newcommand{\Todo}[1]{{\color{blue}#1}}
\newcommand{\Secref}[1]{Section~\ref{#1}}
\newcommand{\Chapref}[1]{Section~\ref{#1}}
\newcommand{\Algoref}[1]{Algorithm~\ref{#1}}
\newcommand{\Table}{\Constraint{Table}~}
\newcommand{\Extensional}{\Constraint{Extensional}~}
\newcommand{\Lineref}[1]{Line~\ref{#1}}
\newcommand{\Linesref}[2]{Lines~\ref{#1}-\ref{#2}}

\newcommand{\Method}[2]{\textbf{method~}\mathrm{{#1}}({#2})}
\newcommand{\MethodReturn}[3]{\textbf{method~}\mathrm{{#1}}({#2})\textbf{\ : \ {#3}}}
\newcommand{\Class}{\textbf{Class~}}
\newcommand{\Constructor}{\textbf{constructor~}}

\newcommand{\Dom}[1]{\text{dom}({#1})}

% SparseBitSet
\newcommand{\Words}{\texttt{words}}
\newcommand{\Index}{\texttt{index}}
\newcommand{\Mask}{\texttt{mask}}
\newcommand{\Limit}{\texttt{limit}}
\newcommand{\SparseBitSet}{\texttt{SparseBitSet}}
\newcommand{\Offset}{\texttt{offset}}

% CT Propagator
\newcommand{\Scp}{\texttt{scp}}
\newcommand{\CurrTable}{\texttt{currTable}}
\newcommand{\Sval}{\texttt{S^{val}}}
\newcommand{\Ssup}{\texttt{S^{sup}}}
\newcommand{\LastSizes}{\texttt{lastSize}}
\newcommand{\Supports}{\texttt{supports}}
\newcommand{\Residues}{\texttt{residues}}

% Pseduo code
\newcommand{\ForEach}[1]{\textbf{foreach } {#1} \textbf{ do }}
\newcommand{\ForEachTo}[3]{\textbf{foreach } {#1} \textbf{ from } {#2} 
  \textbf{ to } {#3} \textbf{ do }}
\newcommand{\ForEachDownTo}[3]{\textbf{foreach } {#1} \textbf{ from } {#2} 
  \textbf{ downto } {#3} \textbf{ do }}
\newcommand{\Break}{\textbf{break~}}
\newcommand{\While}[1]{\textbf{while~} {#1} \textbf{~do~}}

\renewcommand{\algorithmicforall}{\textbf{Method}}
\renewcommand{\algorithmicdo}{}
\renewcommand{\algorithmicwhile}{\textbf{foreach}}

\newcommand{\Func}[2]{\FORALL{#1(#2)}}
\newcommand{\FuncRet}[3]{\FORALL{#1(#2) \ : \ \textbf{#3}}}
\newcommand{\Endfunc}{\ENDFOR}
\newcommand{\To}{~\bf{to}~}
\newcommand{\Downto}{~{\bf{downto}}~}
\newcommand{\For}[3]{\FOR{${#1} \leftarrow {#2} \To {#3}$ \textbf{do}}}
\newcommand{\ForDown}[3]{\FOR{${#1} \leftarrow {#2} \Downto {#3}$ \textbf{do}}}
\newcommand{\FOREACH}[1]{\WHILE{{#1} \textbf{do}}}
\newcommand{\ENDFOREACH}{\ENDWHILE}

\newcommand{\function}[1]{\mathrm{#1}}
\newcommand{\localvar}[1]{\mathit{#1}}

\newlength\myindent
\setlength\myindent{2em}
\newcommand\bindent{%
  \begingroup
  \setlength{\itemindent}{\myindent}
  \addtolength{\algorithmicindent}{\myindent}
}
\newcommand\eindent{\endgroup}

\newcommand{\INDSTATE}[1][1]{\STATE\hspace{#1\algorithmicindent}}
\newcommand{\INDRETURN}[1][1]{\STATE\hspace{#1\algorithmicindent}\textbf{return~}}
\newcommand{\INDIF}[2][1]{\STATE\hspace{#1\algorithmicindent}
  \textbf{if~}{#2}\textbf{~then}}
\newcommand{\INDELSE}[1][1]{\STATE\hspace{#1\algorithmicindent}\textbf{else~}}
\newcommand{\INDELSEIF}[2][1]{\STATE\hspace{#1\algorithmicindent}
  \textbf{else if~}{#2}\textbf{~then}}

\newcommand{\CTpaper}[0]{DBLP:conf/cp/DemeulenaereHLP16}

\numberwithin{equation}{section}

\title{\textbf{Implementation and Evaluation of a\\
    Compact Table Propagator in Gecode
  }
}

\author{Linnea Ingmar} % replace by your name(s)

%\date{Month Day, Year}
\date{\today}

\begin{document}

\maketitle

\tableofcontents

\newpage

\section{Introduction}
\label{intro}

In Constraint Programming (CP), every constraint is associated with a propagator
algorithm. The propagator algorithm filters out impossible values for the variables
related to the constraint. For the \Table constraint, several propagator
algorithms are known. In 2016, a new propagator algorithm for the \Table
constraint was published~\cite{\CTpaper}, called Compact Table (CT).
Preliminary results indicate that CT outperforms the previously known algorithms.
There has been no attempt to implement CT in the constraint solver Gecode~\cite{Gecode}, and consequently its performance in Gecode is unknown.

\subsection{Goal}
\label{intro:goal}
The goal of this thesis is to implement a CT progatator
algorithm for the \Table constraint in Gecode,
and to evaluate its performance with respect to the existing propagators.

\subsection{Contributions}
\label{intro:contributions}

\Todo{State the contributions, perhaps as a bulleted list, referring to the different
parts of the paper, as opposed to giving a traditional outline. (As suggested
by Olle Gallmo.)}

This thesis contributes with the following:

\begin{itemize}
  \item The relevant preliminaries have been covered in \Chapref{bg}.
  \item The algorithms presented in~\cite{DBLP:conf/cp/DemeulenaereHLP16} have been modified to suit the
    target constraint solver Gecode, and are presented and explained in 
    \Chapref{algorithms}.
  \item The CT algorithm has been implemented in Gecode, see \Chapref{implementation}.
  \item The performance of the CT algorithm has been evaluated, see \Chapref{evaluation}.
  \item ...
\end{itemize}

\section{Background}
\label{bg}

% Definiera alla begrepp som används senare

This chapter provides a background that is relevant for the
following chapters. It is divided into three parts: \Secref{bg:cp}
introduces Constraint Programming. \Secref{bg:gecode} gives an overview
of Gecode, a constraint solver. Finally, \Secref{bg:table} introduces
the \Table constraint.

\subsection{Constraint Programming}
\label{bg:cp}
This section introduces the concept of Constraint Programming (CP).

CP is a programming paradigm that is used for solving
combinatorial problems. A problem is
modelled as a set of \emph{constraints} and a
set of \emph{variables} with possible values. The possible values of 
a variable is called the \emph{domain} of the variable.
All the variables are to be assigned a value
from their domains, so that all the constraints of the problem
are satisfied.
%  There need not
% exist a solution to a given problem, and a solution need not
% be unique; zero or more assignments of values to the variables may 
% satisfy the constraints.
One small example is the set of variables~$\Set{x,y}$ with
domain~$\Set{1,2,3}$,
together with the constraint~$x<y$. This problem has two solutions, 
namely~$(x,y) = (1,2), (x,y) = (1,3)$ and~$(x,y) = (2,3)$.
If, on the other hand, the domains of~$x$ and~$y$ would have been~$\Set{2,3}$
and~$\Set{1,2}$, respectively, then there would exist no solution since none of
the possible assignments of~$x$ and~$y$ would have satisfied the constraint~$x<y$.

Sometimes the solution should not only satisfy the set of constraints for the
problem, but should also maximise or minimise some given function.
Take the same small example as above, with
the domains of~$x,y$ being~$\Set{1,2,3}$, and suppose that the solution
should maximise the sum of~$x$ and~$y$. Then the solution is~$(x,y) = (2,3)$.

A constraint solver is a software that solves constraint problems.
The solving of a problem consists of generating a search tree by branching
on possible values for the variables. At each node in the search tree,
the solver removes impossible values from the domains of the variables.
This filtering process is called \emph{propagation}. Each constraint is
associated with at least one propagator algorithm, whose task is to detect
values that would violate the constraint if the variables were to be assigned
any of those values, and remove those values from the domain of the variable.

After this intuitive description we are ready to define some concepts.

% Intuition
% Example
% Definitions
% - CSP

% Propagation
% - Constistency levels (value-, bounds-, domain-)
% 

\paragraph{Propagation}
% From chap14

\begin{definition}
  A domain~$D$ is a complete mapping from a fixed (countable) set of 
  variables~$V$ to finite sets of integers. A domain~$D$ is failed,
  if $D(x) = \emptyset$ for some $x \in V$. A variable~$x \in V$
  is fixed by a domain D, if $|D(x)| = 1$. The intersection of domains~$D1$ and~$D2$,
  denoted~$D1 \sqcap D2$, is defined by the domain
  $D(x) = D1(x) \Union D2(x)$ for all $x \in V$.
  A domain~$D1$ is stronger than a domain~$D2$, written $D1 \sqsubseteq D2$,
  if~$D1(x) \subset D2(x)$ for all $x \in V$.
\end{definition}

A propagator~$p$ is a function that maps domains to domains.

\subsection{Gecode}
\label{bg:gecode}
Gecode~\cite{Gecode} is a popular constraint programming solver written in C++.

% definiera de delar av Gecodes API som dyker upp senare, såsom propagate(), status messages
% använda inbyggda klasen BitSets?

\subsection{The \Table Constraint}
\label{bg:table}
The \Table constraint, called \Extensional in Gecode,
explicitly expresses the possible combinations of values for the variables as a
sequence of $n$-tuples.

\subsection{Compact-Table Propagator}
\label{bg:ct}
% Beskriv huvudidéerna
% Komplexitet? Kolla artikeln om negativa table-villkor
% O(r*d*t) per table constraint along a branch in the search tree (artikeln om bakgrund)

\subsection{Sparse Bit-Set}
\label{bg:sbc}
% Beskriv idén

\section{Algorithms}
\label{algorithms}

% Section 3 bör beskriva din design i detalj men samtidigt inte på C++-nivå. Jag gillar att se sjok av pseudokod inbäddade i text som förklarar pseudokoden. Man kan skriva text mellan sjoken och/eller i caption till algorithm-omgivningen. Något som jag också gillar är stepwise refinement, dvs. att först visa en enkel men korrekt version, och sedan en eller flera mer sofistikerade, optimerade versioner. Den pseudokod som du har skrivit passar bra i Section 3, men bryt gärna upp åtminstone Class CT-Propagator i flera stycken algorithm-omgivningar.

This chapter presents the algorithms that are used in the implementation of the
CT propagator in \Chapref{implementation}. In what follows, when we refer to
an array~$a$,~$a[0]$ denotes the first element (indexing starts from~$0$),
$a$.length() the number of its cells and~$a[i:j]$ all its cells in the closed
interval~$[i,j]$, where~$0 \leq i \leq j \leq a.\function{length}() - 1$.
When we refer to a two-dimensional array~$m$,~$m[i][*]$ denotes
row~$i$ and~$m[*][j]$ column~$j$, seing~$m$ as a matrix.

\subsection{Sparse Bit-Set}
This section describes Class~$\SparseBitSet$, which is the main datastructure
in the CT algorithm for maintaining the supports.~\Algoref{algo:sparse} shows the
pseudo code for Class~$\SparseBitSet$. The rest of this section describes its
fields and methods in detail.

\begin{algorithm}[H]
  \begin{algorithmic}[1]  % comment [1] away to drop the line numbers
        \STATE $\Class$ SparseBitSet
    \item[]
      \STATE $\Words$: array of long \COMMENT{$\Words.\function{length}() = p$} \label{line:sbsfield:start}
      \STATE $\Index$: array of int \COMMENT{$\Index.\function{length}() = p$}
      \STATE $\Limit$: int 
      \STATE $\Mask$: array of long \COMMENT{$\Mask.\function{length}() = p$} \label{line:sbsfield:end}

    \item[]
    \Func{initSparseBitSet}{$\localvar{nbits}$: int} \label{line:initsbs:start}
      \STATE $\localvar{p} \leftarrow \Ceiling{\frac{\localvar{nbits}}{64}}$
      \STATE $\Words \leftarrow \text{~array of long of length~} \localvar{p} \text{, first~} 
      \localvar{nbits} \text{~set to 1}$
      \STATE $\Mask \leftarrow \text{~array of long of length} \localvar{p} \text{, all bits set to~}0$
      \STATE $\Index \leftarrow [0, \ldots, \localvar{p} - 1]$
      \STATE $\Limit \leftarrow \localvar{p-1}$ \label{line:initsbs:end}
      \Endfunc

    \item[]
      \FuncRet{isEmpty}{}{Boolean} \label{line:isEmpty:1}
      \RETURN{$\Limit = -1$} \label{line:isEmpty:2}
      \Endfunc
    \item[]
      \Func{clearMask}{} \label{line:clearMask:1}
      \For{i}{0}{\Limit} \label{line:clearMask:2}
        \STATE $\localvar{offset} \leftarrow \Index[i]$ \label{line:clearMask:3}
        \STATE $\texttt{mask}[\localvar{offset}] \leftarrow 0^{64}$ \label{line:clearMask:4}
      \ENDFOR
      \Endfunc

    \item[]
      \Func{reverseMask}{} \label{line:reverse:1}
      \For{i}{0}{\Limit} \label{line:reverse:2}
      \STATE $\Offset \leftarrow \Index[i]$ \label{line:reverse:3}
      \STATE $\texttt{mask}[\Offset] \leftarrow ~\texttt{mask}[\Offset]$ \COMMENT{birwise NOT} \label{line:reverse:4}
      \ENDFOR
      \Endfunc



    % \item[] 
    %   \Func{reverseMask}{}  \COMMENT{Not currently used in CT algorithm}
    %   \STATE $\ForEachTo{i}{0}{\Limit}$
    %   \STATE $\localvar{offset} \leftarrow \Index[i]$
    %   \STATE $\texttt{mask}[\localvar{offset}] \leftarrow 
    %   {\raise.17ex\hbox{$\scriptstyle\mathtt{\sim}$}}
    %   \texttt{mask}[\localvar{offset}]$ \COMMENT{bitwise NOT}
      %\Endfunc

    \item[]
      \Func{addToMask}{$\localvar{m}$: array of long} \label{line:addToMask:1}
      \For{i}{0}{\Limit} \label{line:addToMask:2}
      \STATE $\localvar{offset} \leftarrow \Index[i]$ \label{line:addToMask:3}
      \STATE $\texttt{mask}[\localvar{offset}] \leftarrow \texttt{mask}[\localvar{offset}] \ | \ 
      \localvar{m}[\localvar{offset}]$ \COMMENT{bitwise OR} \label{line:addToMask:4}
      \ENDFOR
      \Endfunc

    \item[]
      \Func{intersectWithMask}{} \label{line:intersect:1}
      %\FOR{$i \leftarrow \Limit \Downto 0$}
      \ForDown{i}{\Limit}{0} \label{line:intersect:2}
      \STATE $\localvar{offset} \leftarrow \Index[i]$ \label{line:intersect:3}
      \STATE $w \leftarrow \Words[\localvar{offset}] \ \& \ \Mask[\localvar{offset}]$ \COMMENT{bitwise AND} \label{line:intersect:4}
      \IF{$w \neq \Words[\localvar{offset}]$} \label{line:intersect:5}
      \STATE $\Words[\localvar{offset}] \leftarrow w$ \label{line:intersect:6}
      \IF{$w = 0^{64}$} \label{line:intersect:7}
      \STATE $\Index[i] \leftarrow \Index[\Limit]$ \label{line:intersect:8}
      \STATE $\Index[\Limit] \leftarrow \localvar{offset}$ \label{line:intersect:8.5}
      \STATE $\Limit \leftarrow \Limit - 1$ \label{line:intersect:9}
      \ENDIF
      \ENDIF
      \ENDFOR
      \Endfunc

    \item[]
      \FuncRet{intersectIndex}{$\localvar{m}$: array of long}{int} \label{line:interIdx:1}
      \For{i}{0}{\Limit} \label{line:interIdx:2}
      \STATE $\Offset \leftarrow \Index[i]$ \label{line:interIdx:4}
      \IF{$\Words[\Offset] \ \& \ m[\Offset] \neq 0^{64}$} \label{line:interIdx:5}
      \RETURN{$\Offset$} \label{line:interIdx:6}
      \ENDIF
      \ENDFOR
      \RETURN{$-1$} \label{line:interIdx:7}
      \Endfunc


    \end{algorithmic}
  \caption{Pseudo code for Class SparseBitSet.}
  \label{algo:sparse}
\end{algorithm}

\subsubsection{Fields}
\Todo{It should not be neccassary to keep \Mask~as a field, as it is only used temporarily.
Unneccessary to copy it every time.}

\Todo{Todo: Add examples.}

Linesref{line:sbsfield:start}{line:sbsfield:end} of~\Algoref{algo:sparse} shows the fields
of Class~\SparseBitSet~and their types. Now follows a more detailed description of them.

\begin{itemize}
  \item \Words~is an array of~$p$ 64-bit words which defines the current value of the bit-set:
    the~$i$th bit of the~$j$th word is 1 if and only if the $(j-1) \cdot 64 + i$th element of
    the set is present. Initially, all words in this array have all their bits set to~$1$,
    except the last word that may have a suffix of bits set to~$0$. \Todo{Example.}
  \item \Index~is an array that manages the indices of the words in~\Words. The
    indices of all the non-zero words are in~\Index~at positions less than or
    equal to the value of the field~\Limit.
    The indices of the zero-words will be overwritten by indices of non-zero words
    because only non-zero words are relevant. \Todo{In the paper, they switch place
    of the indices of the zero-word and the non-zero word. But since we implement this
    in Gecode, which is copy-based, we
    can overwrite it instead. Explain this difference?}
  \item \Limit~is the largest index of~\Index~corresponding to a non-zero word
    in~\Words.
    \Todo{Not entirely true, because the indices of the non-zero words will be
      still be "lying around" at indices $> \Limit$.
      But the point is that we only \emph{care} about indices 0...\Limit~in~\Index.
    Should think of a better formulation.}
  \item \Mask~is a local temporary array that is used to modify the bits in~\Words.
    
    % collect elements with
    % the method addToMask(). It can be cleared with method clearMask(). 
    % A~\SparseBitSet can only be modified by means of the method intersectWithMask().
\end{itemize}

\noindent
The class invariant describing the state of the class is as follows:
\begin{alignat}{1}
  &\Index[0:\Limit]~\text{is a permutation of a subset of~} [0,\dots,p-1],\text{~and} \\
  &\forall i \in \Set{0,\dots,\Limit}: \Words[\Index[i]] \neq 0^{64}
\end{alignat}

\Todo{In the paper, the invariant looks like this:
\begin{alignat}{1}
  &\Index~\text{is a permutation of~} [0,\dots,p-1],\text{~and} \\
  &\forall i \in \Set{0,\dots,p-1}: i \leq \Limit \Leftrightarrow \Words[\Index[i]] \neq 0^{64}
\end{alignat}

but for us it is different because we overwrite the indices of the zero-words.
}

\subsubsection{Methods}
\Todo{Add method initialiseSparseBitSet() to show the initialisation of the 
  object?}

\subsection{Compact-Table (CT) Algorithm}
This section describes the CT algorithm, a domain consistent propagation
algorithm for any table constraint~$c$.~\Algoref{algo:CT} shows the interface
for Class CT-Propagator, which implements the CT algorithm. The rest of this
section will describe its fields and methods in detail.
\Todo{Is this a good idea or is the pseudo code too spread out? Ugly that the
lines of references "Algorithm x" are numbered in~\Algoref{algo:CT}.}
%It dynamically maintains a set of valid supports regarding the current domain of each variable.

 \begin{algorithm}[H]
  \begin{algorithmic}[1]  % comment [1] away to drop the line numbers
    \STATE $\Class$ CT-Propagator
    \item[]
      \STATE $\Scp$: array of variables
      \STATE $\CurrTable$: SparseBitSet \COMMENT{Current supported tuples}
      \STATE $\Supports$: array of BitSet   \COMMENT{$\Supports[x,a]$ is the bit-set of supports for $(x,a)$}
      \STATE $\LastSizes$: array of int \COMMENT{$\LastSizes[x]$ is the last size of the domain of $x$.c}
      \STATE $\Residues$: array of int  \COMMENT{$\Residues[x,a]$ is the last found support for $(x,a)$. 
      No residues yet!}
            
    \item[]
      \Func{updateTable}{}   
      \FOREACH{$x \in \Scp \text{~such that~} |\Dom{x}| \neq \LastSizes[x]$}
        \STATE $\LastSizes[x] \leftarrow |\Dom{x}|$
        \STATE $\CurrTable$.clearMask() 
        \FOREACH{$a \in \Dom{x}$} %\COMMENT{No incremental update yet!}
          \STATE $\CurrTable$.addToMask($\Supports[x,a]$)
          \STATE $\CurrTable$.intersectWithMask()
          \IF{$\CurrTable$.isEmpty()}
            \STATE $\Break$ \COMMENT{No valid tuples left}
          \ENDIF
        \ENDFOREACH
     \ENDFOREACH
     \Endfunc

    \item[]
      \Func{filterDomains}{}
      \STATE $\localvar{count\_unassigned} \leftarrow 0$
      \FOREACH{$x \in \Scp \text{~such that~} |\Dom{x}| > 1$}
            \FOREACH{$a \in \Dom{x}$}
                  \STATE $\texttt{index} \leftarrow \CurrTable$.intersectIndex($\Supports[x,a]$)
                  \IF{$\texttt{index~} \neq -1$}
                        \STATE \todo{save index in residues} \COMMENT{No residues yet!}
                  \ELSE
                        \STATE $\Dom{x} \leftarrow \Dom{x} \setminus \Set{a}$
                        \IF{$|\Dom{x}| > 1$}
                           \STATE $\localvar{count\_unassigned} \leftarrow \localvar{count\_unassigned} + 1$
                        \ENDIF
                  \ENDIF
             \ENDFOREACH
      \ENDFOREACH
      \IF{$\localvar{count\_unassigned} \leq 1$}
        \RETURN{$Subsumed$}
      \ELSE
        \RETURN{$Fixpoint$}
      \ENDIF
      \Endfunc

    \item[]
      \STATE $\Method{propagate}{}$
      \STATE updateTable()
      \IF{$\CurrTable$.isEmpty()}
        \RETURN{$Failed$}
      \ENDIF
      \RETURN{filterDomains()}

    \end{algorithmic}
  \caption{Interface for CT propagator class.}
  \label{algo:CT}
\end{algorithm}

%This description is mainly taken from~\cite{CTpaper}.

% Beskriv övergripande?

% CT is based on bitwise operations -- among the important datastructures we have
% a~$\SparseBitSet$ object which maintains the valid supports, and also a bitset
% matrix that

% In this ordered set, each tuple is indexed
% by the order it appears in the table, and the~$i$th element is~$1$ if and only if
% the~$i$th tuple is a valid support, else the~$i$th element is~$0$.

\subsubsection{Fields}
\label{CT:fields}

\Todo{Add examples with figures for describing the fields.}

% From hereon, we let the \emph{initial valid table} for $c$,~$T_v$, be a subset of the
% initial table~$T$ such that for all tuples~$\tau \in T_v$, $\tau$ is a
% support on~$c$. \Todo{Define: initial table.}
\Linesref{line:CTfield:start}{line:CTfield:end} of \Algoref{algo:CT}
shows the fields of Class \texttt{CT-Propagator} and their types.
Now follows a more detailed
description of them.

\begin{itemize}
  \item $\Scp$ represents the scope of the constraint~$c$.
  \item $\CurrTable$ represents the current table,
    that is, the valid supports for~$c$. If the initial table of~$c$ is
    $\Tuple{\tau_0, \tau_1, ..., \tau_{p-1}}$, then~$\CurrTable$ is a 
    $\SparseBitSet$ object of initial size~$p$, such that value~$i$
    is contained (is set to~$1$) if and only if the~$i$th tuple is valid:
    
    \begin{equation} \label{eq:currtable}
      i \in \CurrTable \ \Leftrightarrow \ \forall x \in scp(c): \tau_i[x] \in \Dom{x}
    \end{equation}

  \item $\Supports$ represents the supports for each variable-value pair~$(x,a)$,
    where~$x \in scp(c) \land a \in \Dom{x}$.
    It is a static array of words~$\Supports[x,a]$, seen as bit-sets.
    The bit-set~$\Supports[x,a]$ is such that
    the bit at position~$i$ is set to~$1$ if and only if the 
    tuple~$\tau_i$ in the initial table of~$c$ is a support for~$(x,a)$:

    \begin{equation}
      \forall x \in scp(c): \forall a \in \Dom{x}:
      \Supports[x,a][i] = 1 \Leftrightarrow \tau_i[x] = a \ \land \
      \forall y \in scp(c): \tau_i[y] \in \Dom{y}
    \end{equation}

    % One can optimise this definition slighty since it is unnecessary to keep track of
    % invalid supports. Instead, only the tuples that are valid supports for~$c$
    % need to be indexed in~$\Supports[x,a]$, potentially making the bit-set shorter
    % if only a subset of the tuples in the initial table are valid supports. 
    % If the initial table 
    % is~$T = \Tuple{\tau_0, \tau_1, ..., \tau_{p-1}}$ and it turns out that
    % $\tau_0,...,\tau_{i-1}$ are valid supports and~$\tau_{i}$ is invalid, then~$\tau_{i+1}$
    % will correspond to index~$i$ instead of~$i+1$ in~$\Supports[x,a]$. So a better
    % definition reads:

    % \begin{equation}
    %   \forall x \in scp(c): \forall a \in \Dom{x}:
    %   \Supports[x,a][j] = 1 \Leftrightarrow \tau_i[x] = a \ \land \
    %   \forall y \in scp(c): \tau_i[y] \in \Dom{y}
    % \end{equation}
    
    % where~$j = |\Set{\tau_k \ | \ \tau_k \in T \ \land \ \tau_k \text{~is valid} \ \land \ k < i}|$.
    Seing~$\Supports$ as a matrix, we have that the column~$\Supports[*][i]$ encodes
    the~$i$th support for~$c$. $\Supports$ is computed once during the initialisation of CT and then
    remains unchanged.
    
  \item $\LastSizes$ is an array that contains the domain size of each
    variable~$x$ right before the previous invocation of CT on~$c$.
    % Not upon initialisation though

  \item $\Residues$ is an array such that for each variable-value pari~$(x,a)$,
    $\Residues[x,a]$ denotes the index of the word in~$\CurrTable$ where a support
    was found for~$(x,a)$ the last time it was sought for.

\end{itemize}

\subsubsection{Methods}

We now describe the methods of Class \texttt{CT-Propagator}.

\paragraph{Initialisation.}
The initialisation of the fields of Class \texttt{CT-Propagator} is described in
\Algoref{algo:initialise-CT}. The method initialiseCT() takes two parameters:
$\localvar{variables}$, that are the variables associated to the constraint~$c$,
and~$\localvar{tuples}$, that is a list of tuples that define the initial table for~$c$.

\begin{algorithm}[H]
  \begin{algorithmic}[1]  % comment [1] away to drop the line numbers
          \Func{initialiseCT}{$\localvar{variables}, \localvar{tuples}$} 
      \STATE $\localvar{npairs} \leftarrow \function{sum}\Set{|\Dom{x}| : x \in \localvar{variables}}$
      \COMMENT{Number of variable-value pairs}\label{line:init:3}
      \STATE $\localvar{ntuples} \leftarrow \localvar{tuples}.\function{size}()$ \COMMENT{Number of tuples}
      \STATE $\localvar{nsupports} \leftarrow 0$ \COMMENT{Number of found supports} \label{line:init:4}
      
      \STATE $\Scp \leftarrow \localvar{variables}$ \label{line:init:1}
      \STATE $\LastSizes \leftarrow \text{array of  length~} \Scp.\function{length}
      ~\text{filled with -1}$ \COMMENT{Dummy value} \label{line:init:2}
      \STATE $\Residues \leftarrow \text{array of length~} \localvar{npairs}$ \label{line:init:9}
      
      \STATE $\Supports \leftarrow$ \text{array of length~}$\localvar{npairs}$
      \text{with bit-sets of size~}$\localvar{ntuples}$ \label{line:init:5}
      % BitSets$(\localvar{size}, \localvar{ntuples})$ 
      %\COMMENT{bit-set matrix with $\localvar{size}$ rows and $\localvar{ntuples}$ columns} 

      \FOREACH{$t \in \T{tuples}$} \label{line:init:6}
        \STATE $\localvar{supported} \leftarrow \T{true}$
        \FOREACH{$x \in \Scp$}
          \IF{$t[x] \notin \Dom{x}$}
            \STATE $\localvar{supported} \leftarrow \T{false}$
            \STATE $\textbf{break}$ \COMMENT{Exit loop}
          \ENDIF
        \ENDFOREACH
          \IF{$\localvar{supported}$} 
            \STATE $\localvar{nsupports} \leftarrow \localvar{nsupports} + 1$
            \FOREACH{$x \in \Scp$} \label{line:init:9}
              \STATE $\Supports[x,t[x]][\localvar{nsupports}] \leftarrow 1$ \label{line:init:10}
              \STATE $\Residues[x,t[x]] \leftarrow \Floor{\frac{\localvar{nsupports}}{64}}$
              \COMMENT{Index for the support in~$\CurrTable$}\label{line:init:11}
            \ENDFOREACH
            %\STATE setElemsInColumn($\localvar{nsupports}$, $t$) 
          \ENDIF
      \ENDFOREACH \label{line:init:7}
      %\COMMENT{Mark tuple as supported}

     % \STATE $\Supports.\text{trimToWidth}(\localvar{nsupports})$ \COMMENT{Keep only the first $\localvar{nsupports}$ bits for each row}
      \FOREACH{$x \in \Scp$} \label{line:init:12}
          \STATE{$\Dom{x} \leftarrow \Dom{x} \setminus \Set{a \in \Dom{x} : \Supports[x,a] = 0^{64}}$}\label{line:init:14}
      \ENDFOREACH
      \STATE $\CurrTable \leftarrow \SparseBitSet(\localvar{nsupports})$ 
      \COMMENT{$\SparseBitSet$ with $\localvar{nsupports}$ bits} \label{line:init:15}
      \Endfunc

  \end{algorithmic}
  \caption{Pseudo code for initialising the CT-propagator.}
  \label{algo:initialise-CT}
\end{algorithm}


\Linesref{line:init:3}{line:init:4}~initialise local variables that will be 
used later.

\Linesref{line:init:1}{line:init:5}~initialise the fields~\Scp~,~\LastSizes,
~\Residues~and~\Supports.
The array~\LastSizes~is filled with 
the dummy value~$-1$ because we
want to have~$|\Dom{x}| \neq \LastSizes[x]$ the first time that the propagation
algorithm runs. The field \Residues~will be filled with values later, 
in line \ref{line:init:11}.
The field \Supports~is initalised as an array of bit-sets, with one bit-set for each
variable-value pair, and the size of each
bit-set being the number of tuples in~$\localvar{tuples}$. Each bit-set is assumed
to be initially filled with zeros.

\Linesref{line:init:6}{line:init:7} set the correct bits to~$1$ in~$\Supports$.
For each tuple~$t$, we check if~$t$ is a valid support for~$c$. Recall that~$t$ is
a valid support for~$c$ if and only if~$t[x] \in \Dom{x}$ for all~$x \in scp(c)$.
We keep a counter,~$nsupports$, for the number of valid supports for~$c$.
This is used for indexing the tuples in~$\Supports$ (we only index the tuples
that are valid supports).
If~$t$ is a valid support,
all elements in~$\Supports$ corresponding to~$t$ are set to~$1$ in
line \ref{line:init:10}. We also take the opportunity to store the index
of the found support in~$\Residues[x,t[x]]$
in line~\ref{line:init:11}.
\Todo{Perhaps break out lines \ref{line:init:6}-\ref{line:init:7} to an own algorithm-environment?
I haven't implemented residues yet (to keep the first version as simple as possible)
but this is how I would do it.}
%\Todo{trimToWidth()}.

\Lineref{line:init:8} initialises~$\CurrTable$ as a~$\SparseBitSet$ object with
$nsupports$ bits, initially with all bits set to~$1$ since~$nsupports$
number of tuples are initially valid supports for~$c$. \Todo{We assume that 
\SparseBitSet~has a constructor that takes nr of bits as arguments.}

\Linesref{line:init:12}{line:init:14} removes values that are not supported
by any tuple in the initial table. \Todo{Optimisation, not neccassary
  as these values will be removed upon the first propagation otherwise.}

 
  To limit the size of~$\Supports$, one could perform simple bounds
  propagation in initialiseCT(). \Algoref{algo:initial-propagation}
  is inserted in the beginning of initialiseCT(), in line~\ref{line:init:3}
  in~\Algoref{algo:initialise-CT}. It removes
  from the domain of each variable~$x$ all values that are either greater 
  than the largest element or smaller than the smallest element in the
  initial table. \Todo{This is not neccassary, only an optimisation.
  Might change later to take the intersection of the 
  values in the tupleset with the domain of the variables, but this is cheaper
  since it is very cheap to find min and max for tuplesets.}

\begin{algorithm}[H]
  \begin{algorithmic}[1]  % comment [1] away to drop the line numbers
    \FOREACH{$x \in \localvar{variables}$} %\COMMENT{Initial bounds propagation}
        \STATE $\Dom{x} \leftarrow \Dom{x} \setminus  
        \Set{a \in \Dom{x} : a > \localvar{tuples}.\function{max}()
          ~\text{or}~a < \localvar{tuples}.\function{min}()}$
      \ENDFOREACH
  \end{algorithmic}
  \caption{Simple initial propagation for keeping down the size of \Supports,
  to be inserted in the beginning of initialiseCT() in~\Algoref{algo:initialise-CT}.}
  \label{algo:initial-propagation}
\end{algorithm}

\paragraph{Performing propagation.}
When the propagator is invoked for propagation, the method propagate()
in \Algoref{algo:CT} is called. Before defining this function, we need
to define the help functions updateTable() and filterDomains().
Performing propagation consists of two steps: updating the current
table and filtering out inconsistent values from the domains of the variables.
We now describe these processes in more detail.

\begin{enumerate}
\item \textit{Updating the current table.} 
  
  \begin{algorithm}[H]
  \begin{algorithmic}[1]  % comment [1] away to drop the line numbers
       \Func{updateTable}{} \label{line:updateTable:1} 
      \FOREACH{$x \in \Scp \text{~such that~} |\Dom{x}| \neq \LastSizes[x]$} \label{line:updateTable:2} 
        \STATE $\LastSizes[x] \leftarrow |\Dom{x}|$ \label{line:updateTable:3} 
        \STATE $\CurrTable$.clearMask() \label{line:updateTable:4} 
        \FOREACH{$a \in \Dom{x}$} \label{line:updateTable:5} 
          \STATE $\CurrTable$.addToMask($\Supports[x,a]$) \label{line:updateTable:6} 
        \ENDFOREACH      
        \STATE $\CurrTable$.intersectWithMask() \label{line:updateTable:7} 
        \IF{$\CurrTable$.isEmpty()} \label{line:updateTable:8} 
          \STATE $\Break$ \COMMENT{No valid tuples left} \label{line:updateTable:9} 
        \ENDIF
      \ENDFOREACH
      \Endfunc

  \end{algorithmic}
  \caption{Method updateTable() in Class CT-Propagator.}
  \label{algo:updateTable}
\end{algorithm}

  The method updateTable() in 
  \Linesref{line:updateTable:1}{line:updateTable:9} of~\Algoref{algo:CT}
  filters out (indices of)
  tuples that have ceased to be supports since the last invocation of the
  propagator. The ovarall idea is that for each variable~$x \in \Scp$, we keep 
  only those values in~$\CurrTable$ that correspond to valid tuples for~$x$, and the
  complement of those values are set to~$0$. We note that
  it is only necessary to consider a variable~$x$ if its domain has changed
  since the last invocation of the propagator, because only then
  may the set of valid tuples have changed.~\Lineref{line:updateTable:2}
  chooses exactly those variables.

  \Lineref{line:updateTable:3} records the domain size for~$x$ in~$\LastSizes[x]$
  and line~\ref{line:updateTable:4} clears the temporary mask (sets all bits to~$0$).

  \Linesref{line:updateTable:5}{line:updateTable:6} stores the union of the
  set of valid tuples for each value~$a \in \Domain{x}$ in the temporary mask
  and \Lineref{line:update:table:7} intersects~$\CurrTable$ with the mask.
  \Lineref{line:updateTable:8} checks whether the current table is empty,
  in which case we break in line~\ref{line:currTable:9}
  because there are no valid tuples left.

\item 
  \textit{Filtering of domains.}
  After the current table has been updated, inconsistent values must be removed
  from the domains of the variables.   
  It follows from the definition of the bit-sets~$\CurrTable$ and~$\Supports[x,a]$
  that~$(x,a)$ has a valid support if and only if 

  \begin{equation}
    \label{eq:validcond}
    (\CurrTable \Inter \Supports[x,a]) \neq \emptyset
  \end{equation}

  Therefore, we must check this condition for every variable-value pair~$(x,a)$ and
  remove~$a$ from the domain of~$x$ if the condition is not satisfied any more.
  This is implemented in the method filterDomains()
  in~\Algoref{algo:filterDomains}.%lines~\ref{line:filterDom:0}-\ref{line:filterDom:12}.

  \begin{algorithm}[H]
    \begin{algorithmic}[1]  % comment [1] away to drop the line numbers
            \Func{filterDomains}{} \label{line:filterDom:0}
      \STATE $\localvar{count\_unassigned} \leftarrow 0$ \label{line:filterDom:1}
      \FOREACH{$x \in \Scp \text{~such that~} |\Dom{x}| > 1$} \label{line:filterDom:2}
            \FOREACH{$a \in \Dom{x}$} \label{line:filterDom:3}
                  \STATE $\localvar{index} \leftarrow \CurrTable$.intersectIndex($\Supports[x,a]$) \label{line:filterDom:4}
                  \IF{$\localvar{index~} \neq -1$} \label{line:filterDom:5}
                        \STATE \todo{save index in residues} \COMMENT{No residues yet!} \label{line:filterDom:6}
                  \ELSE
                        \STATE $\Dom{x} \leftarrow \Dom{x} \setminus \Set{a}$ \label{line:filterDom:7}
                  \ENDIF
             \ENDFOREACH
             \IF{$|\Dom{x}| > 1$} \label{line:filterDom:8}
                 \STATE $\localvar{count\_unassigned} \leftarrow \localvar{count\_unassigned} + 1$\label{line:filterDom:9}
             \ENDIF

      \ENDFOREACH
      \IF{$\localvar{count\_unassigned} \leq 1$} \label{line:filterDom:10}
        \RETURN{$Subsumed$} \label{line:filterDom:11}
      \ELSE
        \RETURN{$Fixpoint$} \label{line:filterDom:12}
      \ENDIF
      \Endfunc

    \end{algorithmic}
    \caption{Method filterDomains() in Class CT-Propagator.}
    \label{algo:filterDomains}
  \end{algorithm}

  \Todo{This method would be slightly more sofisticated if residues
  were implemented. For now, disregard the empty line in the if-statement.}

  \Lineref{line:filterDom:1} initialises a counter for the number of unassigned
  variables.

  \Linesref{line:filterDom:2}{line:filterDom:9} performs the
  actual filtering of the domains. We note that it is only neccessary to
  consider a variable~$x \in \Scp$ whose domain size is larger than~$1$,
  because we will never filter out values from the domain of an assigned
  varible. To see this, assume we removed the last value for a variable~$x$,
  causing a wipe-out for~$x$. Then by the definition in equation~\eqref{eq:currtable}
  \CurrTable~must be empty,
  which it will not be upon invocation of filterDomains(). Hence, we need
  only consider~$x \in \Scp$ such that~$|\Dom{x} > 1|$.

  In line~\ref{line:filterDom:4}, we search for an index in~$\CurrTable$
  where a valid support for the variable-value pair~$(x,a)$ is found, 
  thereby checking the condition in~\eqref{eq:validcond}.
  In line~\ref{line:filterDom:7} we remove~$a$ from~$\Dom{x}$ if~$(x,a)$
  is not supported any more.

  \Linesref{line:filterDom:8}{line:filterDom:9}
  increments the counter of unassigned variables if~$|\Dom{x}| > 1$.

  \Linesref{line:filterDom:10}{line:filterDom:11} return the correct
  propagator status message. If the number of unassigned variables is
  at most one, the propagator is subsumed. Otherwise, the propagator
  is at fixpoint.
  
\end{enumerate}

After defining updateTable() and filterDomains(), we are now ready to
define the method propagate() in Class CT-Propagator, shown 
in~\Algoref{algo:propagate}.

\begin{algorithm}[H]
  \begin{algorithmic}[1]  % comment [1] away to drop the line numbers
    \Func{propagate}{} \label{line:propagate:1}
                   %              \STATE updateTable() \label{line:propagate:2}
              \IF{$\CurrTable$.isEmpty()} \label{line:propagate:3}
                      \RETURN{$Failed$} \label{line:propagate:4}
              \ENDIF
              \RETURN{filterDomains()} \label{line:propagate:5}
         \Endfunc

  \end{algorithmic}
  \caption{Method propagate() in Class CT-Propagator. updateTable()
    (\Algoref{algo:updateTable}) is
  called, and if the current table is empty, we are in a failed node.
  Otherwise, filterDomains() (\Algoref{algo:filterDomains})
  is called, and the return value of that method is returned.}
  \label{algo:propagate}
\end{algorithm}

\subsubsection{Propagator Status Messages}
A propagtor signals a status message after propagation.
For CT there are three possible status messages;~\emph{Fail},
\emph{Subsumed} and~\emph{Fixpoint}.

\paragraph{Fail.}
A propagator must correctly signal failure
if it has decided that the constraint is unsatisfiable for the input store~$S$.
At the latest a propagator must be
able to decide whether or not a~$S$ is a solution store when all
variables have been assigned.
CT has two ways of detecting a failure: either when all bits in~$\CurrTable$
are set to zero -- meaning that none of the tuples are valid, or when the
size of the domain of a variable is zero.

\paragraph{Subsumption.}
A propagator is not allowed to signal subsumption if it could propagate further
at a later point. At the latest, a propagator must signal subsumption if all the variables
are assigned in~$S$ and a~$S$ is a solution store.
CT signals subsumption when at most one variable is not assigned,
since this is the pointe where no more propagation can be made.

\paragraph{Fixpoint.}
A propagator is not allowed to claim that it has computed a fixpoint if
it could still propagate.
CT always computes a fixpoint if it is not subsumed or wipes out the domain of a variable.
To understand this, consider two consequtive calls to propagate().
Let~$T$ be the set of valid tuples and~$x$ be the set of variables.
The first time propagate() is executed, a (possibly empty) subset~$T_r$ of~$T$
is invalidated (that is, the corresponding bits in $\CurrTable$ are set to $0$)
in updateTable().
Assuming $\CurrTable$ is not empty after the return of updateTable(), which would
cause a backtrack in the search, filterDomains() is called.
This method removes a (possibly empty) set of values~$V_i$ from the domain 
of each variable~$x_i$.
Each value $v_i \in V_i$ has the property that a subset~$T_{v_i}$ 
of the tuples in~$T_r$ are the last supports for~$(x_i,v_i)$. In other words,
$T \setminus T_{v_i}$ does not contain a support for~$(x_i,v_i)$.
So in the second call to updateTable(), no tuples are invalidated,
because none of the tuples in~$T \setminus T_r$ is a support for any variable-value
pair~$(x_i, v_i) \in \Set{x_i} \times V_i, i \in \Set{1...|x|}$.
Hence, the second call to propagate() does not give
any further propagation.

\section{Implementation}
\label{implementation}

% Section 4 blir nog mindre intressant än Section 3 och 5, men där kan du skriva om sånt som är specifikt för C++ och Gecode för att algoritmerna i Section 3 ska fungera, precis som du har börjat göra. Det är också en bra plats för detaljer som sopats under mattan i pseudokoden, t.ex. exakt hur du mappar (x,a) till rätt element i supports och residues, med hashtabell eller så.

This chapter describes an implementation of the CT propagator using the algorithms
presented in \Chapref{algorithms}. The implementation was made in the C++ programming
language in the Gecode library.

The bit-set matrix $\Supports$ is static and can be shared between all solution spaces.

The bit-set $\CurrTable$ changes dynamically during propagation and must therefore be copied for
every new space. Can save memory by only copying the non-zero words.

No need to save the $\T{tuples}$ as a field in the propagator class as all
the necessary information is encoded in $\CurrTable$ and $\Supports$.


% Can't modify the value of the variable while iterating over it
% when using an iterator for a view, the view cannot be modified (or, in C++ lingua: modifying the variable invalidates the iterator).

% The motivation to iterate over range sequences rather than individual values is efficiency:
% as there are typically less ranges than indvidual values, iteration over ranges can be more efficient.

% Sharing of domain and iterators (argument false in inter_v)

%http://www.gecode.org/doc/4.4.0/reference/classGecode_1_1Iter_1_1Values_1_1BitSet.html

% First perform bounds propagation

% Staging p. 324

% Region (memory allocation)

% Multimap for hashing rows?

\section{Evaluation}
\label{evaluation}
This chapter presents the evaluation of the implementation of the CT propagator
presented in \Chapref{implementation}. In \Secref{evaluation:setup},
the evaluation setup is described. In \Secref{evaluation:results} presents
the results of the evaluation. The results are discussed in \Secref{evaluation:discussion}.

\subsection{Evaluation Setup}
\label{evaluation:setup}
\subsection{Results}
\label{evaluation:results}
\subsection{Discussion}
\label{evaluation:discussion}

\section{Conclusions and Future Work}
\label{conclusions}

\bibliographystyle{abbrv}
\bibliography{astra}

\appendix
\section{Source Code}
\label{source code}

This appendix presents the source code for the implementation described in \Chapref{implementation}.

\end{document}

%%% Local Variables:
%%% mode: latex
%%% TeX-master: t
%%% End:

% OscaR source code:
% https://bitbucket.org/oscarlib/oscar/src/40e25aafba8f9b0ab06029449350a2a9d1614854/oscar-algo/src/main/scala/oscar/algo/reversible/ReversibleSparseBitSet.scala?at=dev&fileviewer=file-view-default
% https://bitbucket.org/oscarlib/oscar/src/40e25aafba8f9b0ab06029449350a2a9d1614854/oscar-cp/src/main/scala/oscar/cp/constraints/tables/TableCT.scala?at=dev&fileviewer=file-view-default3

% 