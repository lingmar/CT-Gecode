\documentclass[a4paper,11pt]{article}
\usepackage{fullpage}
\usepackage{booktabs}
\usepackage{tikz}
\usepackage{pgfplots}
\pgfplotsset{plot coordinates/math parser=false}
\usetikzlibrary{calc}
\usepackage{astra}
%\usepackage{etoolbox}\AtBeginEnvironment{algorithmic}{\small‌​}
%\usepackage{algorithm2e}
%\usepackage{algorithmicx}
%\input{macros}

%\usepackage{amsthm}
\usepackage{amsthm}

\newtheorem{definition}{Definition}
%\newtheorem{proof}{Proof}
\newtheorem{theorem}{Theorem}[section]
\newtheorem{corollary}{Corollary}[theorem]
\newtheorem{lemma}[theorem]{Lemma}

\newcommand{\CT}[0]{\emph{CT~}}

% Silly but saves space
\newcommand{\T}[1]{\texttt{#1}}

\newcommand{\Timeout}{600.00} % CPU seconds
\newcommand{\Todo}[1]{{\color{blue}#1}}
\newcommand{\Secref}[1]{Section~\ref{#1}}
\newcommand{\Chapref}[1]{Section~\ref{#1}}
\newcommand{\Algoref}[1]{Algorithm~\ref{#1}}
\newcommand{\Table}{\Constraint{Table}~}
\newcommand{\Extensional}{\Constraint{Extensional}~}
\newcommand{\Lineref}[1]{Line~\ref{#1}}
\newcommand{\Linesref}[2]{Lines~\ref{#1}-\ref{#2}}
\newcommand{\lineref}[1]{line~\ref{#1}}
\newcommand{\linesref}[2]{lines~\ref{#1}-\ref{#2}}
\newcommand{\Defref}[1]{Definition~\ref{#1}}
\newcommand{\Thmref}[1]{Theorem~\ref{#1}}
\newcommand{\Lemmaref}[1]{Lemma~\ref{#1}}


\newcommand{\Eqref}[1]{\eqref{#1}}

\newcommand{\Method}[2]{\textbf{method~}\mathrm{{#1}}({#2})}
\newcommand{\MethodReturn}[3]{\textbf{method~}\mathrm{{#1}}({#2})\textbf{\ : \ {#3}}}
\newcommand{\Class}{\textbf{Class~}}
\newcommand{\Constructor}{\textbf{constructor~}}

\newcommand{\Dom}[1]{\text{dom}({#1})}
\newcommand{\Dominit}[1]{\underline{\text{dom}}(#1)}


%\newcommand{\Ceiling}[1]{\left\lceil#1\right\rceil}
%\newcommand{\Floor}[1]{\left\lfloor#1\right\rfloor}


% SparseBitSet
\newcommand{\Words}{\texttt{words}}
\newcommand{\Index}{\texttt{index}}
\newcommand{\Mask}{\texttt{mask}}
\newcommand{\Limit}{\texttt{limit}}
\newcommand{\SparseBitSet}{\texttt{SparseBitSet}}
\newcommand{\Offset}{\texttt{offset}}

% CT Propagator
\newcommand{\Scp}{\texttt{vars}}
\newcommand{\CurrTable}{\texttt{validTuples}}
\newcommand{\Sval}{\texttt{S^{val}}}
\newcommand{\Ssup}{\texttt{S^{sup}}}
\newcommand{\LastSizes}{\texttt{lastSize}}
\newcommand{\Supports}{\texttt{supports}}
\newcommand{\Residues}{\texttt{residues}}

% Pseduo code
\newcommand{\ForEach}[1]{\textbf{foreach } {#1} \textbf{ do }}
\newcommand{\ForEachTo}[3]{\textbf{foreach } {#1} \textbf{ from } {#2} 
  \textbf{ to } {#3} \textbf{ do }}
\newcommand{\ForEachDownTo}[3]{\textbf{foreach } {#1} \textbf{ from } {#2} 
  \textbf{ downto } {#3} \textbf{ do }}
\newcommand{\Break}{\textbf{break~}}
\newcommand{\While}[1]{\textbf{while~} {#1} \textbf{~do~}}

\renewcommand{\algorithmicforall}{\textbf{Method}}
\renewcommand{\algorithmicdo}{}
\renewcommand{\algorithmicwhile}{\textbf{foreach}}

\newcommand{\Func}[2]{\FORALL{#1(#2)}}
\newcommand{\FuncRet}[3]{\FORALL{#1(#2) \ : \ \textbf{#3}}}
\newcommand{\Endfunc}{\ENDFOR}
\newcommand{\To}{~\bf{to}~}
\newcommand{\Downto}{~{\bf{downto}}~}
\newcommand{\For}[3]{\FOR{${#1} \leftarrow {#2} \To {#3}$ \textbf{do}}}
\newcommand{\ForDown}[3]{\FOR{${#1} \leftarrow {#2} \Downto {#3}$ \textbf{do}}}
\newcommand{\FOREACH}[1]{\WHILE{{#1} \textbf{do}}}
\newcommand{\ENDFOREACH}{\ENDWHILE}

\newcommand{\function}[1]{\mathrm{#1}}
\newcommand{\localvar}[1]{\mathit{#1}}

\newlength\myindent
\setlength\myindent{2em}
\newcommand\bindent{%
  \begingroup
  \setlength{\itemindent}{\myindent}
  \addtolength{\algorithmicindent}{\myindent}
}
\newcommand\eindent{\endgroup}

\newcommand{\INDSTATE}[1][1]{\STATE\hspace{#1\algorithmicindent}}
\newcommand{\INDRETURN}[1][1]{\STATE\hspace{#1\algorithmicindent}\textbf{return~}}
\newcommand{\INDIF}[2][1]{\STATE\hspace{#1\algorithmicindent}
  \textbf{if~}{#2}\textbf{~then}}
\newcommand{\INDELSE}[1][1]{\STATE\hspace{#1\algorithmicindent}\textbf{else~}}
\newcommand{\INDELSEIF}[2][1]{\STATE\hspace{#1\algorithmicindent}
  \textbf{else if~}{#2}\textbf{~then}}

\newcommand{\CTpaper}[0]{DBLP:conf/cp/DemeulenaereHLP16}

\numberwithin{equation}{section}

\title{\textbf{Implementation and Evaluation of a\\
    Compact Table Propagator in Gecode
  }
}

\author{Linnea Ingmar} % replace by your name(s)

%\date{Month Day, Year}
\date{\today}

\begin{document}

\maketitle

\tableofcontents

\newpage

\section{Introduction}
\label{intro}

In Constraint Programming (CP), every constraint is associated with a propagator
algorithm. The propagator algorithm filters out impossible values for the variables
related to the constraint. For the \Table constraint, several propagator
algorithms are known. In 2016, a new propagator algorithm for the \Table
constraint was published~\cite{\CTpaper}, called Compact Table (CT).
Preliminary results indicate that CT outperforms the previously known algorithms.
There has been no attempt to implement CT in the constraint solver Gecode~\cite{Gecode}, and consequently its performance in Gecode is unknown.

\subsection{Goal}
\label{intro:goal}
The goal of this thesis is to implement a CT propagator
algorithm for the \Table constraint in Gecode,
and to evaluate its performance with respect to the existing propagators.

\subsection{Contributions}
\label{intro:contributions}

\Todo{State the contributions, perhaps as a bulleted list, referring to the different
parts of the paper, as opposed to giving a traditional outline. (As suggested
by Olle Gallmo.)}

This thesis contributes with the following:

\begin{itemize}
  \item The relevant preliminaries have been covered in \Chapref{bg}.
  \item The algorithms presented in~\cite{DBLP:conf/cp/DemeulenaereHLP16} have been modified to suit the
    target constraint solver Gecode, and are presented and explained in 
    \Chapref{algorithms}.
  \item The CT algorithm has been implemented in Gecode, see \Chapref{sec:implementation}.
  \item The performance of the CT algorithm has been evaluated, see \Chapref{evaluation}.
  \item ...
\end{itemize}

\section{Background}
\label{bg}

% Definiera alla begrepp som används senare

This section provides a background that is relevant for the
following sections. It is divided into five parts: \Secref{bg:cp}
introduces Constraint Programming. \Secref{bg:gecode} gives an overview
of Gecode, a constraint solver. \Secref{bg:table} introduces the~\Table
constraint. \Secref{bg:ct} describes the main concepts of the Compact
Table (CT) algorithm. Finally, \Secref{bg:sbs} describes the main
idea of \Todo{Reversible?} Sparse Bit-Sets,
a data structure that is used in the CT algorithm.

\subsection{Constraint Programming}
\label{bg:cp}
This section introduces the concept of Constraint Programming (CP).

CP is a programming paradigm that is used for solving
combinatorial problems. A problem is
modelled as a set of \emph{constraints} and a
set of \emph{variables} with possible values. The possible values of 
a variable is called the \emph{domain} of the variable.
All the variables are to be assigned a value
from their domains, so that all the constraints of the problem
are satisfied. Sometimes the solution should not only satisfy the set of constraints for the
problem, but should also maximise or minimise some given function.

\Todo{Perhaps add an example somewhere here.}

%  There need not
% exist a solution to a given problem, and a solution need not
% be unique; zero or more assignments of values to the variables may 
% satisfy the constraints.
% One small example is the set of variables~$\Set{x,y}$ with
% domain~$\Set{1,2,3}$,
% together with the constraint~$x<y$. This problem has two solutions, 
% namely~$(x,y) = (1,2), (x,y) = (1,3)$ and~$(x,y) = (2,3)$.
% If, on the other hand, the domains of~$x$ and~$y$ would have been~$\Set{2,3}$
% and~$\Set{1,2}$, respectively, then there would exist no solution since none of
% the possible assignments of~$x$ and~$y$ would have satisfied the constraint~$x<y$.


% Take the same small example as above, with
% the domains of~$x,y$ being~$\Set{1,2,3}$, and suppose that the solution
% should maximise the sum of~$x$ and~$y$. Then the solution is~$(x,y) = (2,3)$.

A constraint solver (CP solver) is a software that solves constraint problems.
The solving of a problem consists of generating a search tree by branching
on possible values for the variables. At each node in the search tree,
the solver removes impossible values from the domains of the variables.
This filtering process is called \emph{propagation}. Each constraint is
associated with at least one propagator algorithm, whose task is to detect
values that would violate the constraint if the variables were to be assigned
any of those values, and remove those values from the domain of the variables.

% Intuition
% Example
% Definitions
% - CSP

% Propagation
% - Constistency levels (value-, bounds-, domain-)
% 

\begin{definition}
  \textbf{Constraint.} Consider a finite sequence of~$n$ 
  variables~$X = x_1,\ldots,x_n$, and a corresponding sequence of
  \emph{domains}~$D = D_1,\ldots,D_n$, that are possible values for the
  respective variable. 
  For a variable~$x_i \in X$, its domain~$D_i$ is denoted 
  by~$dom(x_i)$.
  \begin{itemize}
    \item   A \emph{constraint}~$c$ on~$X$ is a relation, 
      denoted by~$rel(c)$. The associated variables~$X$ are denoted~$vars(c)$,
      and we call~$|vars(c)|$ the \emph{arity} of~$c$. The relation
      $rel(c)$ contains the set of~$n$-tuples that are allowed
      for~$X$, we call those~$n$-tuples \emph{solutions} to the constraint~$c$.
    \item   For an~$n$-tuple~$\tau = \Tuple{a_1,\ldots,a_n}$ associated with~$X$, we
      denote the~$i$th value of~$\tau$ by~$\tau[i]$ or~$\tau[x_i]$. The 
      tuple~$\tau$ is \emph{valid} for~$X$
      if and only if each value of~$\tau$ is in the domain of the corresponding
      variable: $\forall i \in 1 \ldots n, \tau[i] \in dom(x_i)$, or equivalently,
      $\tau \in D_1 \times \ldots \times D_n$.
    \item An~$n$-tuple~$\tau$ is a \emph{support} on a~$n$-ary constraint~$c$ if and only
      if~$\tau$ is valid for~$vars(c)$ and~$\tau$ is a solution to~$c$, that is,
      $\tau$ is contained in~$rel(c)$.
    \item For an~$n$-ary constraint~$c$, involving a variable~$x$ such that
      the value~$a \in dom(x)$, an~$n$-tuple~$\tau$ is a 
      \emph{support for}~$(x,a)$ on~$c$ if and only if~$\tau$ is a support on~$c$,
      and~$\tau[x] = a$.
    \end{itemize}
\end{definition}

\begin{definition}
  \textbf{CSP.} A Constraint Satisfaction Problem (CSP) is a 
  triple~$\left<V,D,C\right>$, where:
  $V = v_1, \ldots, v_n$ is a finite sequence of variables,
  ~$D = D_1, \ldots, D_n$ is a finite sequence of domains for the respective variable,
  and~$C = \Set{c_1, \ldots, c_m}$ is a set of constraints, each on a subsequence of~$V$.
\end{definition}

\begin{definition}
  \textbf{Stores.} A \emph{store}~$s$ is a function, mapping a finite set of
  variables~$V = v_1, \ldots, v_n$ to a finite set of domains. We denote the domain of
  a variable~$v_i$ under~$s$ by~$s(v_i)$ or~$\Dom{v_i}$.
  \begin{itemize}
    \item A store~$s$ is \emph{failed} if and only if~$s(v_i) = \varnothing$ for some~$v_i \in V$.
    \item   A variable~$v_i \in V$ is \emph{fixed}, or \emph{assigned},
      by a store~$s$ if and only if~$|s(v_i)| = 1$. 
    \item A store~$s$ is an \emph{assignment} store if all variables are 
      fixed under~$s$.

    \item Let~$c$ be an $n$-ary constraint on~$V$. A store~$s$ is 
      a \emph{solution store} 
      to~$c$ if and only if~$s$ is an assignment store and the
      corresponding~$n$-tuple is a solution to~$c$:
      $\forall i \in \Set{1,\ldots,n}, s(v_i) = \Set{a_i}$,
      and~$\left<a_1,\ldots,a_n\right>$ is a solution to~$c$.

    \item A store~$s_1$ is \emph{stronger} than a store~$s_2$, 
      written~$s_1 \preceq s_2$ if and only if~$s_1(v) \subseteq s_2(v)$ 
      for all~$v \in V$.
  \end{itemize}

\end{definition}

\subsection{Propagation and propagators}

Constraint propagation is the process of removing values from the domains
of the variables in a CSP that can never appear in a solution store to the 
CSP. In a CP solver, each constraint that the solver implements is associated with 
one or more propagation algorithms, called propagators, whose task is to remove
values that are in conflict with its respective constraint.

To have a well-defined behaviour of propagators, there are some properties that
they must fulfill. Now follows a definition of a propagator and the obligations
that they must meet, taken from \Todo{Chapter 14 in handbook} and \Todo{MPG}.

\begin{definition} \label{def:prop}
  \textbf{Propagators.} A \emph{propagator}~$p$ is a function mapping stores to stores:
  \begin{equation*}
    p: store \mapsto store
  \end{equation*}

  In a CP-solver, a propagator is implemented as a procedure that also returns 
  a \emph{status message}. A propagator must fulfill the following properties:

  \begin{itemize}
  \item A propagator~$p$ is a decreasing function:~$p(s) \preceq s$ for any store~$s$.
    This property guarantees that constraint propagation only removes values.

  \item A propagator~$p$ is a monotonic function:
    ~$s_1 \preceq s_2 \Rightarrow p(s_1) \preceq p(s_2)$
    for any stores~$s_1$ and~$s_2$. This property guarantees that constraint propagation
    preserves the strength-ordering of stores.

  \item A propagator is correct for the constraint it implements.
    A propagator~$p$
    is correct for a constraint~$c$ if and only if it does not
    remove values that are part of supports for~$c$.
    This property guarantees that a propagator does not miss any potential 
    solution store.

  \item A propagator is \emph{checking}: for a given assignment store~$s$, the propagator
    must decide whether~$s$ is a solution store or not for the constraint it
    implements. If~$s$ is a solution store, it must signal subsumption, otherwise
    it must signal failure.

  \item A propagator must be \emph{honest}: it must be 
    \emph{fixpoint honest} and \emph{subsumption honest}. 
    A propagator~$p$ is fixpoint honest if and only if it does not signal 
    fixpoint if does not return a fixpoint, and it is subsumption honest
    if and only if it does
    not signal subsumption if it is not subsumed by an input store~$s$.
    
    A propagator~$p$ is at \emph{fixpoint} on a store~$s$ if and only if applying 
    $p$ to to~$s$ gives no further propagation:~$p(s) = s$ for
    a store~$s$. If a propagator~$p$ always returns a fixpoint, that is, 
    if~$p(s) = p(p(s))$, $p$ is \emph{idempotent}.

    A propagator is \emph{subsumed} by a store~$s$ if and only if
    all stronger stores are fixpoints:~$\forall s'\preceq s,p(s')=s$.

\end{itemize}

\end{definition}
Note that the honest property of a propagator does not mean that a
propagator is obliged to signal fixpoint or subsumption
if it has computed a fixpoint or is subsumed, only that it may not 
claim that it is at fixpoint or is subsumed if it is not. 
Thus, it is always safe 
(though in many cases not so efficient for the CP-solver)
for a propagator to signal 'not fixpoint', except for a solution store
when it must signal either fail or subsumption.
In fact, a propagator must not even prune values. An extreme case is
the identity propagator~$i$, with~$i(s) = s$ for all input stores~$s$,
which would be correct
for all constraints, as long at is it checking and honest.

To give a measure of how strong the constraint propagation of a propagator
is, it is common to declare a \emph{consistency level} of a propagator.
There are three commonly used consistency levels,
\textbf{value consistency, bounds consistency}, and \textbf{domain consistency}.

\begin{definition}
  \textbf{Domain consistency.} A constraint~$c$ is domain consistent on a store~$s$ 
  if and only if for all variable-value
  pair~$(x,a)$ such that~$x \in vars(c)$ and~$a \in dom(x)$, there exists
  at least one support for~$(x,a)$ on~$c$. 
  A propagator~$p$ is domain consistent, iff~$c$ is domain 
  consistent on $p(s)$ for all stores~$s$ such that~$p(s)$ is not a failed store.
\end{definition}

\Todo{Bounds consistency, Value consistency}.

\subsection{Gecode}
\label{bg:gecode}
Gecode~\cite{Gecode} is a popular constraint programming solver written in C++.

\Todo{Define the parts of the Gecode API that is used later (propagate(), status
  messages...)}

% definiera de delar av Gecodes API som dyker upp senare, såsom propagate(), status messages
% använda inbyggda klasen BitSets?
%Här bör du bl.a. skriva allt som är relevant för resten av rapporten om Gecodes API. T.ex. de tre returvärdena som propagerare ska returnera, ungefär som du har skrivit i 3.2.3, fast utan det CT-specifika.

\subsection{The \Table Constraint}
\label{bg:table}
The \Table constraint, also called \Extensional,
explicitly expresses the possible combinations of values for the variables as a
sequence of $n$-tuples.

\begin{definition}
  \textbf{Table constraints.} A
  (positive\footnote{There are also negative table constraints, that list the forbidden tuples instead of the allowed tuples.})
  \emph{table constraint c} is a
  constraint such that~$rel(c)$ is defined explicitly by listing all the
  tuples that are solutions to~$c$.
\end{definition}

\subsection{Compact-Table Propagator}
\label{bg:ct}
% Beskriv huvudidéerna
% Komplexitet? Kolla artikeln om negativa table-villkor
% O(r*d*t) per table constraint along a branch in the search tree (artikeln om bakgrund)

\subsection{Sparse Bit-Set}
\label{bg:sbs}
% Beskriv idén

\section{Algorithms}
\label{algorithms}

% Section 3 bör beskriva din design i detalj men samtidigt inte på C++-nivå. Jag gillar att se sjok av pseudokod inbäddade i text som förklarar pseudokoden. Man kan skriva text mellan sjoken och/eller i caption till algorithm-omgivningen. Något som jag också gillar är stepwise refinement, dvs. att först visa en enkel men korrekt version, och sedan en eller flera mer sofistikerade, optimerade versioner. Den pseudokod som du har skrivit passar bra i Section 3, men bryt gärna upp åtminstone Class CT-Propagator i flera stycken algorithm-omgivningar.

This chapter presents the algorithms that are used in the implementation of the
CT propagator in \Chapref{sec:implementation}. In what follows, when we refer to
an array~$a$,~$a[0]$ denotes the first element (indexing starts from~$0$),
$a$.length() the number of its cells and~$a[i:j]$ all its cells in the closed
interval~$[i,j]$, where~$0 \leq i \leq j \leq a.\function{length}() - 1$.
% When we refer to a two-dimensional array~$m$,~$m[i][*]$ denotes
% row~$i$ and~$m[*][j]$ column~$j$, seeing~$m$ as a matrix.

\subsection{Sparse Bit-Set}
\label{sec:sbs}
This section describes Class~$\SparseBitSet$, which is the main data-structure
in the CT algorithm for maintaining the supports.~\Algoref{algo:sparse} shows the
pseudo code for Class~$\SparseBitSet$. The rest of this section describes its
fields and methods in detail.

\begin{algorithm}[H]
  \begin{algorithmic}[1]  % comment [1] away to drop the line numbers
        \STATE $\Class$ SparseBitSet
    \item[]
      \STATE $\Words$: array of long \COMMENT{$\Words.\function{length}() = p$} \label{line:sbsfield:start}
      \STATE $\Index$: array of int \COMMENT{$\Index.\function{length}() = p$}
      \STATE $\Limit$: int 
      \STATE $\Mask$: array of long \COMMENT{$\Mask.\function{length}() = p$} \label{line:sbsfield:end}

    \item[]
    \Func{initSparseBitSet}{$\localvar{nbits}$: int} \label{line:initsbs:start}
      \STATE $\localvar{p} \leftarrow \Ceiling{\frac{\localvar{nbits}}{64}}$
      \STATE $\Words \leftarrow \text{~array of long of length~} \localvar{p} \text{, first~} 
      \localvar{nbits} \text{~set to 1}$
      \STATE $\Mask \leftarrow \text{~array of long of length} \localvar{p} \text{, all bits set to~}0$
      \STATE $\Index \leftarrow [0, \ldots, \localvar{p} - 1]$
      \STATE $\Limit \leftarrow \localvar{p-1}$ \label{line:initsbs:end}
      \Endfunc

    \item[]
      \FuncRet{isEmpty}{}{Boolean} \label{line:isEmpty:1}
      \RETURN{$\Limit = -1$} \label{line:isEmpty:2}
      \Endfunc
    \item[]
      \Func{clearMask}{} \label{line:clearMask:1}
      \For{i}{0}{\Limit} \label{line:clearMask:2}
        \STATE $\localvar{offset} \leftarrow \Index[i]$ \label{line:clearMask:3}
        \STATE $\texttt{mask}[\localvar{offset}] \leftarrow 0^{64}$ \label{line:clearMask:4}
      \ENDFOR
      \Endfunc

    \item[]
      \Func{reverseMask}{} \label{line:reverse:1}
      \For{i}{0}{\Limit} \label{line:reverse:2}
      \STATE $\Offset \leftarrow \Index[i]$ \label{line:reverse:3}
      \STATE $\texttt{mask}[\Offset] \leftarrow ~\texttt{mask}[\Offset]$ \COMMENT{birwise NOT} \label{line:reverse:4}
      \ENDFOR
      \Endfunc



    % \item[] 
    %   \Func{reverseMask}{}  \COMMENT{Not currently used in CT algorithm}
    %   \STATE $\ForEachTo{i}{0}{\Limit}$
    %   \STATE $\localvar{offset} \leftarrow \Index[i]$
    %   \STATE $\texttt{mask}[\localvar{offset}] \leftarrow 
    %   {\raise.17ex\hbox{$\scriptstyle\mathtt{\sim}$}}
    %   \texttt{mask}[\localvar{offset}]$ \COMMENT{bitwise NOT}
      %\Endfunc

    \item[]
      \Func{addToMask}{$\localvar{m}$: array of long} \label{line:addToMask:1}
      \For{i}{0}{\Limit} \label{line:addToMask:2}
      \STATE $\localvar{offset} \leftarrow \Index[i]$ \label{line:addToMask:3}
      \STATE $\texttt{mask}[\localvar{offset}] \leftarrow \texttt{mask}[\localvar{offset}] \ | \ 
      \localvar{m}[\localvar{offset}]$ \COMMENT{bitwise OR} \label{line:addToMask:4}
      \ENDFOR
      \Endfunc

    \item[]
      \Func{intersectWithMask}{} \label{line:intersect:1}
      %\FOR{$i \leftarrow \Limit \Downto 0$}
      \ForDown{i}{\Limit}{0} \label{line:intersect:2}
      \STATE $\localvar{offset} \leftarrow \Index[i]$ \label{line:intersect:3}
      \STATE $w \leftarrow \Words[\localvar{offset}] \ \& \ \Mask[\localvar{offset}]$ \COMMENT{bitwise AND} \label{line:intersect:4}
      \IF{$w \neq \Words[\localvar{offset}]$} \label{line:intersect:5}
      \STATE $\Words[\localvar{offset}] \leftarrow w$ \label{line:intersect:6}
      \IF{$w = 0^{64}$} \label{line:intersect:7}
      \STATE $\Index[i] \leftarrow \Index[\Limit]$ \label{line:intersect:8}
      \STATE $\Index[\Limit] \leftarrow \localvar{offset}$ \label{line:intersect:8.5}
      \STATE $\Limit \leftarrow \Limit - 1$ \label{line:intersect:9}
      \ENDIF
      \ENDIF
      \ENDFOR
      \Endfunc

    \item[]
      \FuncRet{intersectIndex}{$\localvar{m}$: array of long}{int} \label{line:interIdx:1}
      \For{i}{0}{\Limit} \label{line:interIdx:2}
      \STATE $\Offset \leftarrow \Index[i]$ \label{line:interIdx:4}
      \IF{$\Words[\Offset] \ \& \ m[\Offset] \neq 0^{64}$} \label{line:interIdx:5}
      \RETURN{$\Offset$} \label{line:interIdx:6}
      \ENDIF
      \ENDFOR
      \RETURN{$-1$} \label{line:interIdx:7}
      \Endfunc


    \end{algorithmic}
  \caption{Pseudo code for Class SparseBitSet.}
  \label{algo:sparse}
\end{algorithm}

\subsubsection{Fields}
\label{sbs:fields}
% \Todo{It should not be neccassary to keep \Mask~as a field, as it is only used temporarily.
% Unneccessary to copy it every time.}

\Todo{Todo: Add examples.}

\Linesref{line:sbsfield:start}{line:sbsfield:end} of~\Algoref{algo:sparse} shows the fields
of Class~\SparseBitSet~and their types. Now follows a more detailed description of them.

\begin{itemize}
  \item \Words~is an array of~$p$ 64-bit words which defines the current value of the bit-set:
    the~$i$th bit of the~$j$th word is 1 if and only if the $(j-1) \cdot 64 + i$th element of
    the set is present. Initially, all words in this array have all their bits set to~$1$,
    except the last word that may have a suffix of bits set to~$0$. \Todo{Example.}

  \item \Index~is an array that manages the indices of the words in~\Words,
    making it possible to performing operations to non-zero words only.
    In~\Index, the
    indices of all the non-zero words are at positions less than or
    equal to the value of the field~\Limit, and the indices of the zero-words are
    at indices strictly greater than~\Limit. 

  \item \Limit~is the index of~\Index~corresponding to the last non-zero word in~\Words.
    Thus it is one smaller than the number of non-zero words in~\Words.

    % \Limit~is the largest index of~\Index~corresponding to a non-zero word
    % in~\Words.
    % \Todo{Not entirely true, because the indices of the non-zero words will
    %   still be "lying around" at indices $> \Limit$.
    %   But the point is that we only \emph{care} about indices 0..\Limit~in~\Index.
    % Should think of a better formulation.}
  \item \Mask~is a local temporary array that is used to modify the bits in~\Words.
    
    % collect elements with
    % the method addToMask(). It can be cleared with method clearMask(). 
    % A~\SparseBitSet can only be modified by means of the method intersectWithMask().
\end{itemize}

\noindent
The class invariant describing the state of the class is as follows:

\begin{alignat}{1}
  \label{eq:invariant}
  &\Index~\text{is a permutation of~} [0,\dots,p-1],\text{~and} \\
  &\forall i \in \Set{0,\dots,p-1}: i \leq \Limit \Leftrightarrow \Words[\Index[i]] \neq 0^{64}
\end{alignat}

%\begin{alignat}{1}
%   &\Index[0:\Limit]~\text{is a permutation of a subset of~} [0,\dots,p-1],\text{~and} \\
%   &\forall i \in \Set{0,\dots,\Limit}: \Words[\Index[i]] \neq 0^{64}
% \end{alignat}

\subsubsection{Methods}
We now describe the methods in Class~\SparseBitSet~in~\Algoref{algo:sparse}.

\begin{itemize}
  \item initSparseBitSet() in~\linesref{line:initsbs:start}{line:initsbs:end}
    initialises a sparse bit-set-object. It takes 
    the number of bits as an argument and initialises the fields
    described in~\ref{sbs:fields} in a straightforward way.

  \item isEmpty() in lines~\ref{line:isEmpty:1}-\ref{line:isEmpty:2} checks
    if the number of non-zero words is different from zero. If the limit is
    set to~$-1$, that means that all words are zero-words and the bit-set
    is empty.

  \item clearMask() in lines~\ref{line:clearMask:1}-\ref{line:clearMask:4}
    clears the temporary mask. This means setting to~$0$ all words of~$\Mask$
    corresponding to non-zero words of~\Words.

  \item addToMask() in~\linesref{line:addToMask:1}{line:addToMask:4} collects
    elements to the temporary mask by applying a word-by-word logical bit-wise
    \emph{or} operation with a given bit-set (array of long).
    Once again, this operation is only applied to indices corresponding to
    non-zero words in~\Words.

  \item intersectWithMask() in~\linesref{line:intersect:1}{line:intersect:9}
    considers each non-zero word of~\Words~in turn
    and replaces it by its intersection with the corresponding word of~\Mask.
    In case the resulting new word is~$0$, it (its index) is switched
    by (the index of) the last non-zero word, and~\Limit~is
    decreased by one.
    
    In~\Secref{sec:implementation} we will see that the implementation
    actually can skip~\lineref{line:intersect:8.5} because it is unneccesary
    to save the index of a zero-word in a copy-based solver such as Gecode.
    We keep this
    line here though, because otherwise the invariant in~\Eqref{eq:invariant} 
    would not hold.
    
  \item intersectIndex() in~\linesref{line:interIdx:1}{line:interIdx:7}
    checks whether the intersection of~\Words~and a given bit-set
    (array of long) is empty or not. For all non-zero words in~\Words,
    we perform a logical bit-wise \emph{and} operation 
    in line~\ref{line:interIdx:5} and return
    the index of the word if the intersection is non-empty. If the
    intersection is empty for all words,~$-1$ is returned.
\end{itemize}

\subsection{Compact-Table (CT) Algorithm}
\label{sec:ct}
This section describes the CT algorithm, a domain consistent propagation
algorithm for any \Table constraint~$c$.~\Algoref{algo:CT} shows the interface
for Class CT-Propagator, which implements the CT algorithm. The rest of this
section will describe its fields and methods in detail.

%Ugly that the lines of references "Algorithm x" are numbered in~\Algoref{algo:CT}.}
%It dynamically maintains a set of valid supports regarding the current domain of each variable.

 \begin{algorithm}[H]
  \begin{algorithmic}[1]  % comment [1] away to drop the line numbers
    \STATE $\Class$ CT-Propagator
    \item[]
      \STATE $\Scp$: array of variables
      \STATE $\CurrTable$: SparseBitSet \COMMENT{Current supported tuples}
      \STATE $\Supports$: array of BitSet   \COMMENT{$\Supports[x,a]$ is the bit-set of supports for $(x,a)$}
      \STATE $\LastSizes$: array of int \COMMENT{$\LastSizes[x]$ is the last size of the domain of $x$.c}
      \STATE $\Residues$: array of int  \COMMENT{$\Residues[x,a]$ is the last found support for $(x,a)$. 
      No residues yet!}
            
    \item[]
      \Func{updateTable}{}   
      \FOREACH{$x \in \Scp \text{~such that~} |\Dom{x}| \neq \LastSizes[x]$}
        \STATE $\LastSizes[x] \leftarrow |\Dom{x}|$
        \STATE $\CurrTable$.clearMask() 
        \FOREACH{$a \in \Dom{x}$} %\COMMENT{No incremental update yet!}
          \STATE $\CurrTable$.addToMask($\Supports[x,a]$)
          \STATE $\CurrTable$.intersectWithMask()
          \IF{$\CurrTable$.isEmpty()}
            \STATE $\Break$ \COMMENT{No valid tuples left}
          \ENDIF
        \ENDFOREACH
     \ENDFOREACH
     \Endfunc

    \item[]
      \Func{filterDomains}{}
      \STATE $\localvar{count\_unassigned} \leftarrow 0$
      \FOREACH{$x \in \Scp \text{~such that~} |\Dom{x}| > 1$}
            \FOREACH{$a \in \Dom{x}$}
                  \STATE $\texttt{index} \leftarrow \CurrTable$.intersectIndex($\Supports[x,a]$)
                  \IF{$\texttt{index~} \neq -1$}
                        \STATE \todo{save index in residues} \COMMENT{No residues yet!}
                  \ELSE
                        \STATE $\Dom{x} \leftarrow \Dom{x} \setminus \Set{a}$
                        \IF{$|\Dom{x}| > 1$}
                           \STATE $\localvar{count\_unassigned} \leftarrow \localvar{count\_unassigned} + 1$
                        \ENDIF
                  \ENDIF
             \ENDFOREACH
      \ENDFOREACH
      \IF{$\localvar{count\_unassigned} \leq 1$}
        \RETURN{$Subsumed$}
      \ELSE
        \RETURN{$Fixpoint$}
      \ENDIF
      \Endfunc

    \item[]
      \STATE $\Method{propagate}{}$
      \STATE updateTable()
      \IF{$\CurrTable$.isEmpty()}
        \RETURN{$Failed$}
      \ENDIF
      \RETURN{filterDomains()}

    \end{algorithmic}
  \caption{Interface for CT propagator class.}
  \label{algo:CT}
\end{algorithm}

%This description is mainly taken from~\cite{CTpaper}.

% Beskriv övergripande?

% CT is based on bitwise operations -- among the important datastructures we have
% a~$\SparseBitSet$ object which maintains the valid supports, and also a bitset
% matrix that

% In this ordered set, each tuple is indexed
% by the order it appears in the table, and the~$i$th element is~$1$ if and only if
% the~$i$th tuple is a valid support, else the~$i$th element is~$0$.

\subsubsection{Fields}
\label{CT:fields}

\Todo{Add examples with figures for describing the fields.}

% From hereon, we let the \emph{initial valid table} for $c$,~$T_v$, be a subset of the
% initial table~$T$ such that for all tuples~$\tau \in T_v$, $\tau$ is a
% support on~$c$. \Todo{Define: initial table.}
\Linesref{line:CTfield:start}{line:CTfield:end} of \Algoref{algo:CT}
shows the fields of Class \texttt{CT-Propagator} and their types.
Now follows a more detailed description of them. In what follows, we let
the~\emph{initial domain} of a variable~$x \in \Scp(c)$, denoted~$\Dominit{x}$,
be the domain that~$x$ has before CT has performed any propagation,
in contrast to~$\Dom{x}$ which is the current domain of~$x$.
The \emph{initial table} for a table constraint~$c$ is the list of tuples
$T_0 = \Tuple{\tau_0, \tau_1, \ldots, \tau_{p_0-1}}$ of length~$p_o$
that are given as input to initCT(), and the
\emph{initial valid table} for~$c$ is the subset $T \subseteq T_0$ of size~$p \leq p_0$
such that~$\forall i \in \Set{1, \ldots, p_0}: \tau_i \in T$ iff $\tau_i$ 
is a support on~$c$ for the initial domains of the variables. 

\begin{itemize}
  \item $\Scp$ represents~$vars(c)$, the variables associated with~$c$.
  
  \item $\CurrTable$ represents the current table,
    that is, the current valid supports for~$c$. If the initial valid table for~$c$
    is
    $\Tuple{\tau_0, \tau_1, \ldots, \tau_{p-1}}$,
    then~$\CurrTable$ is a 
    $\SparseBitSet$ object of initial size~$p$, such that value~$i$
    is contained (is set to~$1$) if and only if the~$i$th tuple is valid:
    
    \begin{equation} \label{eq:currtable}
      i \in \CurrTable \ \Leftrightarrow \ \forall x \in vars(c): \tau_i[x] \in \Dom{x}
    \end{equation}

  \item $\Supports$ represents the supports for each variable-value pair~$(x,a)$,
    where~$x \in vars(c) \land a \in \Dom{x}$.
    It is a static array of words~$\Supports[x,a]$, seen as bit-sets.
    The bit-set~$\Supports[x,a]$ is such that
    the bit at position~$i$ is set to~$1$ if and only if the 
    tuple~$\tau_i$ in the initial valid table of~$c$ is initially a support for~$(x,a)$:

    \begin{equation}
      \forall x \in vars(c): \forall a \in \Dom{x}:
      \Supports[x,a][i] = 1 \Leftrightarrow \tau_i[x] = a \ \land \
      \forall y \in vars(c): \tau_i[y] \in \Dominit{y}
    \end{equation}

    % One can optimise this definition slighty since it is unnecessary to keep track of
    % invalid supports. Instead, only the tuples that are valid supports for~$c$
    % need to be indexed in~$\Supports[x,a]$, potentially making the bit-set shorter
    % if only a subset of the tuples in the initial table are valid supports. 
    % If the initial table 
    % is~$T = \Tuple{\tau_0, \tau_1, ..., \tau_{p-1}}$ and it turns out that
    % $\tau_0,...,\tau_{i-1}$ are valid supports and~$\tau_{i}$ is invalid, then~$\tau_{i+1}$
    % will correspond to index~$i$ instead of~$i+1$ in~$\Supports[x,a]$. So a better
    % definition reads:

    % \begin{equation}
    %   \forall x \in vars(c): \forall a \in \Dom{x}:
    %   \Supports[x,a][j] = 1 \Leftrightarrow \tau_i[x] = a \ \land \
    %   \forall y \in vars(c): \tau_i[y] \in \Dom{y}
    % \end{equation}
    
    % where~$j = |\Set{\tau_k \ | \ \tau_k \in T \ \land \ \tau_k \text{~is valid} \ \land \ k < i}|$.
    %Seeing~$\Supports$ as a matrix, we have that the column~$\Supports[*][i]$ encodes
    %the~$i$th support for~$c$.
    $\Supports$ is computed once during the initialisation of CT and then
    remains unchanged.
    
  % \item $\LastSizes$ is an array that contains the domain size of each
  %   variable~$x$ right after the previous invocation of CT on~$c$.
  %   % Not upon initialisation though

  \item $\Residues$ is an array such that for each variable-value pair~$(x,a)$,
    $\Residues[x,a]$ denotes the index of the word in~$\CurrTable$ where a support
    was found for~$(x,a)$ the last time it was sought for.

\end{itemize}

\subsubsection{Methods}

We now describe the methods of Class \texttt{CT-Propagator}.

\paragraph{Initialisation.}
The initialisation of the fields of Class \texttt{CT-Propagator} is described in
\Algoref{algo:initialise-CT}. The method initialiseCT() takes two parameters:
$\localvar{variables}$, that are the variables associated to the constraint~$c$,
and~$\localvar{tuples}$, that is a list of tuples that define the initial table for~$c$.

\begin{algorithm}[H]
  \begin{algorithmic}[1]  % comment [1] away to drop the line numbers
          \Func{initialiseCT}{$\localvar{variables}, \localvar{tuples}$} 
      \STATE $\localvar{npairs} \leftarrow \function{sum}\Set{|\Dom{x}| : x \in \localvar{variables}}$
      \COMMENT{Number of variable-value pairs}\label{line:init:3}
      \STATE $\localvar{ntuples} \leftarrow \localvar{tuples}.\function{size}()$ \COMMENT{Number of tuples}
      \STATE $\localvar{nsupports} \leftarrow 0$ \COMMENT{Number of found supports} \label{line:init:4}
      
      \STATE $\Scp \leftarrow \localvar{variables}$ \label{line:init:1}
      \STATE $\LastSizes \leftarrow \text{array of  length~} \Scp.\function{length}
      ~\text{filled with -1}$ \COMMENT{Dummy value} \label{line:init:2}
      \STATE $\Residues \leftarrow \text{array of length~} \localvar{npairs}$ \label{line:init:9}
      
      \STATE $\Supports \leftarrow$ \text{array of length~}$\localvar{npairs}$
      \text{with bit-sets of size~}$\localvar{ntuples}$ \label{line:init:5}
      % BitSets$(\localvar{size}, \localvar{ntuples})$ 
      %\COMMENT{bit-set matrix with $\localvar{size}$ rows and $\localvar{ntuples}$ columns} 

      \FOREACH{$t \in \T{tuples}$} \label{line:init:6}
        \STATE $\localvar{supported} \leftarrow \T{true}$
        \FOREACH{$x \in \Scp$}
          \IF{$t[x] \notin \Dom{x}$}
            \STATE $\localvar{supported} \leftarrow \T{false}$
            \STATE $\textbf{break}$ \COMMENT{Exit loop}
          \ENDIF
        \ENDFOREACH
          \IF{$\localvar{supported}$} 
            \STATE $\localvar{nsupports} \leftarrow \localvar{nsupports} + 1$
            \FOREACH{$x \in \Scp$} \label{line:init:9}
              \STATE $\Supports[x,t[x]][\localvar{nsupports}] \leftarrow 1$ \label{line:init:10}
              \STATE $\Residues[x,t[x]] \leftarrow \Floor{\frac{\localvar{nsupports}}{64}}$
              \COMMENT{Index for the support in~$\CurrTable$}\label{line:init:11}
            \ENDFOREACH
            %\STATE setElemsInColumn($\localvar{nsupports}$, $t$) 
          \ENDIF
      \ENDFOREACH \label{line:init:7}
      %\COMMENT{Mark tuple as supported}

     % \STATE $\Supports.\text{trimToWidth}(\localvar{nsupports})$ \COMMENT{Keep only the first $\localvar{nsupports}$ bits for each row}
      \FOREACH{$x \in \Scp$} \label{line:init:12}
          \STATE{$\Dom{x} \leftarrow \Dom{x} \setminus \Set{a \in \Dom{x} : \Supports[x,a] = 0^{64}}$}\label{line:init:14}
      \ENDFOREACH
      \STATE $\CurrTable \leftarrow \SparseBitSet(\localvar{nsupports})$ 
      \COMMENT{$\SparseBitSet$ with $\localvar{nsupports}$ bits} \label{line:init:15}
      \Endfunc

  \end{algorithmic}
  \caption{Pseudo code for initialising the CT-propagator.}
  \label{algo:initialise-CT}
\end{algorithm}

\Linesref{line:init:-1}{line:init:0} performs simple bounds
  propagation to limit the domain sizes of the variables,
  which limit the sizes of the incremental data structures.
  It removes
  from the domain of each variable~$x$ all values that are either greater 
  than the largest element or smaller than the smallest element in the
  initial table. If a variable has a domain wipe-out,~$Failed$ is returned.

\Linesref{line:init:3}{line:init:4}~initialise local variables that will be 
used later.

\Linesref{line:init:1}{line:init:5}~initialise the fields~\Scp,
~\Residues~and~\Supports.
The field \Supports~is initialised as an array of bit-sets, with one bit-set for each
variable-value pair, and the size of each
bit-set being the number of tuples in~$\localvar{tuples}$. Each bit-set is assumed
to be initially filled with zeros.

\Linesref{line:init:6}{line:init:7} set the correct bits to~$1$ in~$\Supports$.
For each tuple~$t$, we check if~$t$ is a valid support for~$c$. Recall that~$t$ is
a valid support for~$c$ if and only if~$t[x] \in \Dom{x}$ for all~$x \in scp(c)$.
We keep a counter,~$nsupports$, for the number of valid supports for~$c$.
This is used for indexing the tuples in~$\Supports$ (we only index the tuples
that are valid supports).
If~$t$ is a valid support,
all elements in~$\Supports$ corresponding to~$t$ are set to~$1$ in
line \ref{line:init:10}. We also take the opportunity to store the word index
of the found support in~$\Residues[x,t[x]]$
in line~\ref{line:init:11}.

\Linesref{line:init:12}{line:init:14} removes values that are not supported
by any tuple in the initial valid table. In case a variable has a wipe-out
of its domain,~$Failed$ is returned.

\Lineref{line:init:15} initialises~$\CurrTable$ as a~$\SparseBitSet$ object with
$nsupports$ bits, initially with all bits set to~$1$ since~$nsupports$
number of tuples are initially valid supports for~$c$.~$\localvar{nsupports} > 0$,
otherwise we would have returned~$Failed$ as no variable-value pair would be
supported.

\paragraph{Performing propagation.}
When the propagator is invoked for propagation, the method propagate()
in \Algoref{algo:CT} is called. Before defining this function, we need
to define the help functions updateTable() and filterDomains().
Performing propagation consists of two steps: updating the current
table and filtering out inconsistent values from the domains of the variables.
We now describe these processes in more detail.

\begin{enumerate}
\item \textit{Updating the current table.} 
  
  \begin{algorithm}[H]
  \begin{algorithmic}[1]  % comment [1] away to drop the line numbers
       \Func{updateTable}{} \label{line:updateTable:1} 
      \FOREACH{$x \in \Scp \text{~such that~} |\Dom{x}| \neq \LastSizes[x]$} \label{line:updateTable:2} 
        \STATE $\LastSizes[x] \leftarrow |\Dom{x}|$ \label{line:updateTable:3} 
        \STATE $\CurrTable$.clearMask() \label{line:updateTable:4} 
        \FOREACH{$a \in \Dom{x}$} \label{line:updateTable:5} 
          \STATE $\CurrTable$.addToMask($\Supports[x,a]$) \label{line:updateTable:6} 
        \ENDFOREACH      
        \STATE $\CurrTable$.intersectWithMask() \label{line:updateTable:7} 
        \IF{$\CurrTable$.isEmpty()} \label{line:updateTable:8} 
          \STATE $\Break$ \COMMENT{No valid tuples left} \label{line:updateTable:9} 
        \ENDIF
      \ENDFOREACH
      \Endfunc

  \end{algorithmic}
  \caption{Method updateTable() in Class CT-Propagator. The infrastructure
  is such that this method is called for each variable whose domain is
  modified since the previous call to propagate().}
  \label{algo:updateTable}
\end{algorithm}

  The method updateTable() in~\Algoref{algo:updateTable}
  filters out (indices of)
  tuples that have ceased to be supports for the input variable~$x$.
  \Linesref{line:updateTable:5}{line:updateTable:6} stores the union of the
  set of valid tuples for each value~$a \in \Domain{x}$ in the temporary mask
  and \Lineref{line:updateTable:7} intersects~$\CurrTable$ with the mask,
  so that the indices that correspond to tuples that are no longer valid
  are set to~$0$ in the bit-set.
  \Lineref{line:updateTable:8} checks whether the current table is empty,
  in which case we return~$Failed$ in line~\ref{line:updateTable:9}
  because there are no valid tuples left. 

  The algorithm is assumed to be run on an infrastructure that runs updateTable()
  before every call to propagate(),
  for each variable~$x \in vars(c)$ whose domain has changed since the previous call to
  propagate().

\item 
  \textit{Filtering of domains.}
  After the current table has been updated, inconsistent values must be removed
  from the domains of the variables.   
  It follows from the definition of the bit-sets~$\CurrTable$ and~$\Supports[x,a]$
  that~$(x,a)$ has a valid support if and only if 

  \begin{equation}
    \label{eq:validcond}
    (\CurrTable \Inter \Supports[x,a]) \neq \emptyset
  \end{equation}

  Therefore, we must check this condition for every variable-value pair~$(x,a)$ and
  remove~$a$ from the domain of~$x$ if the condition is not satisfied any more.
  This is implemented in the method filterDomains()
  in~\Algoref{algo:filterDomains}.%lines~\ref{line:filterDom:0}-\ref{line:filterDom:12}.

  \begin{algorithm}[H]
    \begin{algorithmic}[1]  % comment [1] away to drop the line numbers
            \Func{filterDomains}{} \label{line:filterDom:0}
      \STATE $\localvar{count\_unassigned} \leftarrow 0$ \label{line:filterDom:1}
      \FOREACH{$x \in \Scp \text{~such that~} |\Dom{x}| > 1$} \label{line:filterDom:2}
            \FOREACH{$a \in \Dom{x}$} \label{line:filterDom:3}
                  \STATE $\localvar{index} \leftarrow \CurrTable$.intersectIndex($\Supports[x,a]$) \label{line:filterDom:4}
                  \IF{$\localvar{index~} \neq -1$} \label{line:filterDom:5}
                        \STATE \todo{save index in residues} \COMMENT{No residues yet!} \label{line:filterDom:6}
                  \ELSE
                        \STATE $\Dom{x} \leftarrow \Dom{x} \setminus \Set{a}$ \label{line:filterDom:7}
                  \ENDIF
             \ENDFOREACH
             \IF{$|\Dom{x}| > 1$} \label{line:filterDom:8}
                 \STATE $\localvar{count\_unassigned} \leftarrow \localvar{count\_unassigned} + 1$\label{line:filterDom:9}
             \ENDIF

      \ENDFOREACH
      \IF{$\localvar{count\_unassigned} \leq 1$} \label{line:filterDom:10}
        \RETURN{$Subsumed$} \label{line:filterDom:11}
      \ELSE
        \RETURN{$Fixpoint$} \label{line:filterDom:12}
      \ENDIF
      \Endfunc

    \end{algorithmic}
    \caption{Method filterDomains() in Class CT-Propagator.}
    \label{algo:filterDomains}
  \end{algorithm}

  \Lineref{line:filterDom:1} initialises a counter for the number of unassigned
  variables.

  \Linesref{line:filterDom:2}{line:filterDom:9} performs the
  actual filtering of the domains. We note that it is only necessary to
  consider a variable~$x \in \Scp$ whose domain size is larger than~$1$,
  because we will never filter out values from the domain of an assigned
  variable. To see this, assume we removed the last value for a variable~$x$,
  causing a wipe-out for~$x$. Then by the definition in equation~\eqref{eq:currtable}
  \CurrTable~must be empty,
  which it will not be upon invocation of filterDomains(). Hence, we need
  only consider~$x \in \Scp$ such that~$|\Dom{x} > 1|$.

  In \Linesref{line:filterDom:res1}{line:filterDom:res2} we see if the
  cached word index still has a support for~$(x,a)$. It it has not, we
  we search for an index in line~\ref{line:filterDom:4}in~$\CurrTable$
  where a valid support for the variable-value pair~$(x,a)$ is found, 
  thereby checking the condition in~\eqref{eq:validcond}.
  If such an index exists, we cache it in~$\Residues[x,a]$, and
  if it does not, we remove~$a$ from~$\Dom{x}$ if~$(x,a)$ in 
  line~\ref{line:filterDom:7} since there is no support left for~$(x,a)$.

  \Linesref{line:filterDom:8}{line:filterDom:9}
  increments the counter of unassigned variables if~$|\Dom{x}| > 1$.

  \Linesref{line:filterDom:10}{line:filterDom:11} return the correct
  propagator status message. If the number of unassigned variables is
  at most one, the propagator is subsumed. Otherwise, the propagator
  is at fixpoint.
  
\end{enumerate}

After defining updateTable() and filterDomains(), we are now ready to
define the method propagate() in Class CT-Propagator, shown 
in~\Algoref{algo:propagate}.

\Todo{It should be unneccessary to check if validTuples is empty
  as that is done in updateTable already. However, when I try to
  remove the check in the c++ code it crashes, maybe because of
  synchronisation issues between advise() and propagate().}

\begin{algorithm}[H]
  \begin{algorithmic}[1]  % comment [1] away to drop the line numbers
    \Func{propagate}{} \label{line:propagate:1}
                   %              \STATE updateTable() \label{line:propagate:2}
              \IF{$\CurrTable$.isEmpty()} \label{line:propagate:3}
                      \RETURN{$Failed$} \label{line:propagate:4}
              \ENDIF
              \RETURN{filterDomains()} \label{line:propagate:5}
         \Endfunc

  \end{algorithmic}
  \caption{Method propagate() in Class CT-Propagator. updateTable()
    (\Algoref{algo:updateTable}) is
  called, and if the current table is empty, we are in a failed node.
  Otherwise, filterDomains() (\Algoref{algo:filterDomains})
  is called, and the return value of that method is returned.}
  \label{algo:propagate}
\end{algorithm}

\paragraph{Optimisation of propagate().} If~$x$ is the only variable
that has been modified since the last invocation of~$CT$, it is
not necessary to attempt to filter out values from~$x$, because
every value of of~$x$ will have a support in~$\CurrTable$. 
\Todo{Show the implementation of this explicitly?}
 
\subsubsection{Proof of properties for CT}
This section proves that the CT Propagator is indeed a well-defined propagator
implementing the~\Table~constraint. We formulate the following theorem, which
we will prove by a number of lemmas.

\begin{theorem} \label{thm:prop}
  CT is an idempotent, domain consistent propagator implementing 
  the~\Table~constraint, fulfilling the properties in~\Defref{def:prop}.
\end{theorem}

To prove~\Thmref{thm:prop}, we formulate and prove the following lemmas.
In what follows, we denote~$CT(s)$ the resulting store of running 
either \T{initCT()} or \texttt{propagate()} on an input store~$s$,
depending on if it is the first time or not that the propagator is called.

\begin{lemma} \label{lemma:decreasing}
  CT is a decreasing function.
\end{lemma}

\begin{proof}[Proof of \Lemmaref{lemma:decreasing}]
  Since $CT$ only removes values from
  the domains of the variables, we have $CT(s) \preceq s$ for any store $s$.
  Thus, $CT$ is a decreasing function.
\end{proof}

\begin{lemma}\label{lemma:idempotent}
  CT is idempotent.
\end{lemma}

\begin{proof}[Proof of \Lemmaref{lemma:idempotent}]
  To prove that $CT$ is idempotent, we shall show that $CT$ always reaches
  fixpoint for any input store~$s$, that is, $CT(CT(s)) = CT(s)$ for any
  store~$s$.

  Suppose $CT(CT(s)) \neq CT(s)$ for a store~$s$. 
  Since CT is monotonic
  and decreasing, we must have $CT(CT(s)) \prec CT(s)$, that is, $CT$
  must prune at least one value~$a$ from a variable~$x$ from the 
  store~$CT(s)$. 

  By \Eqref{eq:validcond}, there must exists at least one 
  tuple~$\tau_i$
  that is a support for~$(x,a)$ under the store $CT(s)$: 
  $\exists i: i \in \CurrTable \ \land \ \tau_i[x] = a$.
  After \T{updateTable()} is perfomed on~$CT(s)$, we still have
  ~$i \in \CurrTable$, because~$\tau_i$ is still valid in~$CT(s)$.
  Since~\T{filterDomains()} only removes values that have no supports,
  it is impossible that~$a$ is pruned from~$x$, since~$\tau_i$ is a
  support for~$(x,a)$. Hence, we must have~$CT(CT(s)) = CT(s)$.
\end{proof}

% \begin{proof}
%   CT can only remove values from the domains of the variables, it cannot
%   add values to the domains. Therefore, CT is a decreasing function.
% \end{proof}

\begin{lemma}\label{lemma:correct}
  CT is correct for the \Table constraint.
\end{lemma}

\begin{proof}[Proof of \Lemmaref{lemma:correct}]
  $CT$ does not remove values that participate in tuples that are supports
  on a \Table constratint~$c$,
  since \T{filterDomains()} and \T{initCT()} only removes values that 
  have no supports on~$c$. Thus,~$CT$ is correct for \Table.
\end{proof}

\begin{lemma}\label{lemma:checking}
  CT is checking.
\end{lemma}

\begin{proof}[Proof of \Lemmaref{lemma:checking}]
  For an input store~$s$ that is an assignment store, we shall show that $CT$
  signals failure if $s$ is not a solution store, and signals subsumption if
  $s$ is a solution store. 

  First, assume that~$s$ is not a solution store. That means that the tuple
  $\tau = \Tuple{s(x_1),\ldots,s(s_n)}~\notin~c$.
 
  There are two cases, either
  it is the first time $CT$ is applied or it has been applied before.
  If it is the first time, then \T{initCT()} is called.
  Since $\tau$ is not a solution to~$c$, there is at least one variable-value
  pair~$(x_i,s(x_i))$ which is not supported, so~$s(x_i)$ will be pruned
  from~$x$ in \T{initCT()}, which reports failure in line \Todo{which line}.
  
  If it is not the first time that $CT$ is called, \T{propagate()} is called.
  Since there are no valid tuples left, \CurrTable~will be empty after
  the call to \T{updateTable()} and $CT$ reports failure.
  
  Now assume that~$s$ is a solution store. 
  $CT$ signals failure in \T{filterDomains()} because all variables are assigned.
  \Todo{Initialisation?}
\end{proof}

\begin{lemma}\label{lemma:honest}
  CT is honest.
\end{lemma}

\begin{proof}[Proof of \Lemmaref{lemma:honest}]
  Since $CT$ is idempotent, $CT$ is fixpoint honest. It remains to show that
  $CT$ is subsumption honest.~$CT$ signals subsumption on input store~$s$
  if there is at most one
  unassigned variable~$x$ in~\T{filterDomains()}. After this point, no values will
  ever be pruned from~$x$ by~$CT$, because there will always be a support for
  $(x,a)$ for each value~$a \in dom(x)$. Hence,~$CT$ is indeed subsumed by~$s$
  when it signals subsumption.
    
\end{proof}

\begin{lemma}\label{lemma:domain-consistent}
  CT is domain consistent.
\end{lemma}

\begin{proof}[Proof of \Lemmaref{lemma:domain-consistent}]
  There are two cases; either it is the first time~$CT$ is called, or it
  is not. Both after a call to \T{initCT()} and \T{filterDomains()}, 
  for each variable-value
  pair~$(x,a)$ there exists at least one support, because we filter out those
  values that have no support.
\end{proof}

\begin{lemma}\label{lemma:monotonic}
  CT is a monotonic function.
\end{lemma}

\begin{proof}[Proof of \Lemmaref{lemma:monotonic}]
  Consider two stores~$s_1$ and~$s_2$ such that~$s_1 \preceq s_2$.
  Since~$CT$ is domain consistent, each variable-value pair $(x,a)$
  that is part of~$CT(s_1)$, must also be part of~$CT(s_2)$,
  so~$CT(s_1) \preceq CT(s_2)$.
\end{proof}


After proving Lemmas~\ref{lemma:decreasing}-\ref{lemma:monotonic},
proving~\Thmref{thm:prop} is trivial.

\begin{proof}[Proof of \Thmref{thm:prop}]
  The result follows by Lemmas~\ref{lemma:decreasing}-
  \ref{lemma:monotonic}.
\end{proof}

\section{Implementation}
\label{sec:implementation}

\Todo{Copy-function.}



% Section 4 blir nog mindre intressant än Section 3 och 5, men där kan du skriva om sånt som är specifikt för C++ och Gecode för att algoritmerna i Section 3 ska fungera, precis som du har börjat göra. Det är också en bra plats för detaljer som sopats under mattan i pseudokoden, t.ex. exakt hur du mappar (x,a) till rätt element i supports och residues, med hashtabell eller så.

This section describes an implementation of the CT propagator using the algorithms
presented in \Chapref{algorithms}. The implementation was made in the C++ programming
language in the Gecode library.

The bit-set matrix $\Supports$ is static and could be shared between all solution spaces.

The bit-set $\CurrTable$ changes dynamically during propagation and must therefore be copied for
every new space. Can save memory by only copying the non-zero words.

No need to save the $\T{tuples}$ as a field in the propagator class as all
the necessary information is encoded in $\CurrTable$ and $\Supports$.

\Todo{How is the index mapping done in supports and residues?}

% Can't modify the value of the variable while iterating over it
% when using an iterator for a view, the view cannot be modified (or, in C++ lingua: modifying the variable invalidates the iterator).

% The motivation to iterate over range sequences rather than individual values is efficiency:
% as there are typically less ranges than indvidual values, iteration over ranges can be more efficient.

% Sharing of domain and iterators (argument false in inter_v)

%http://www.gecode.org/doc/4.4.0/reference/classGecode_1_1Iter_1_1Values_1_1BitSet.html

% First perform bounds propagation

% Staging p. 324

% Region (memory allocation)

% Multimap for hashing rows?

\section{Evaluation}
\label{evaluation}
This chapter presents the evaluation of the implementation of the CT propagator
presented in \Chapref{sec:implementation}. In \Secref{evaluation:setup},
the evaluation setup is described. In \Secref{evaluation:results} presents
the results of the evaluation. The results are discussed in \Secref{evaluation:discussion}.

% Benchmarks: performance beroende på antalet variabler. 2-ställiga, 3-ställiga, ..., n-ställiga

\subsection{Evaluation Setup}
\label{evaluation:setup}
\subsection{Results}
\label{evaluation:results}

% \begin{tikzpicture}[scale=1.0]
%   \begin{axis}[xmode=log, ymode=log, xmin=1000, ymin=1000, xmax=200000, ymax=200000, xlabel={Ch (ms)}, ylabel={Ge (ms)}]
%     \draw[line width=0.1mm] (axis cs:1000,1000) -- (axis cs:200000,200000);
%     \addplot[scatter, only marks, scatter src=\thisrow{class}, scatter/classes={0={mark=o}}]
%     table[x=ch,y=ge] {plot.dat};
%   \end{axis}
% \end{tikzpicture}

\begin{tikzpicture}[scale=1.0]
  \begin{axis}[
    xmode=log,
    ymin=0,ymax=72,
    xmin=0, xmax=1000000,
    every axis plot/.style={very thick},
    xlabel={timeout limit (ms)},
    ylabel={\# solved},
    legend pos=north west
    % table/create on use/cumulative distribution/.style={
    %   create col/expr={\pgfmathaccuma + \thisrow{f(x)}}   
    % }
    ]
    \addplot [mark=o,green] table {reg.data};
    \addplot [blue] table {tup_mem.data};
    \addplot [yellow] table {tup_speed.data};
    \addplot [red] table {ct.data};
    \legend{Reg,Tup\_mem,Tup\_speed,CT}
  \end{axis}
\end{tikzpicture}

% https://gist.github.com/albins/43a7cf0238a14173e036f0d5f3eb3427

% cat out.log | grep tup | cut -d "(" -f2 | cut -d ")" -f1 | wc -l


\begin{table}[t] \tiny
  \centering
  \begin{tabular}{rrrrrrr}  % right alignment --> decimal point alignment
    $n$ & $runtime_g$ & $fail_g$ & $nprops_g$ & $runtime_c$ & $fail_c$ & $nprops_c$ \\
    \midrule
    0 & 0.119 & 7 & 4679 & 0.366 & 6 & 4474 \\
1 & 0.085 & 2 & 1780 & 0.166 & 11 & 5119 \\
2 & 0.073 & 1 & 1684 & 0.197 & 0 & 1241 \\
3 & 0.052 & 6 & 2065 & 0.137 & 6 & 2266 \\
4 & 0.033 & 0 & 845 & 0.064 & 0 & 850 \\
5 & 0.087 & 0 & 1275 & 0.169 & 0 & 1260 \\
6 & 0.082 & 7 & 3952 & 0.136 & 6 & 3967 \\
7 & 0.077 & 3 & 2304 & 0.102 & 4 & 2460 \\
8 & 0.051 & 2 & 1281 & 0.103 & 2 & 1306 \\
9 & 0.034 & 0 & 825 & 0.072 & 0 & 875 \\
10 & 1.227 & 16 & 171162 & 1.112 & 46 & 298509 \\
11 & 1.399 & 41 & 410228 & 1.063 & 21 & 230481 \\
12 & 1.067 & 3 & 95731 & 1.217 & 71 & 405968 \\
13 & 2.342 & 381 & 1700606 & \Timeout & 147028 & 603399614 \\
14 & 0.978 & 25 & 191179 & 0.873 & 10 & 124704 \\
15 & 3.154 & 245 & 1146764 & 2.368 & 178 & 879023 \\
16 & 2.910 & 185 & 992421 & 1.710 & 152 & 734316 \\
17 & 1.006 & 11 & 136377 & 1.119 & 16 & 128781 \\
18 & 1.052 & 7 & 108977 & 0.958 & 11 & 122557 \\
19 & 3.012 & 166 & 1083832 & 2.131 & 219 & 1381581 \\
20 & 2.690 & 32 & 319505 & 1.858 & 21 & 332253 \\
21 & 2.818 & 61 & 766182 & 5.233 & 1091 & 8287668 \\
22 & \Timeout & 28601 & 171635604 & \Timeout & 69906 & 464113700 \\
23 & 2.475 & 53 & 616660 & \Timeout & 61878 & 414187753 \\
24 & 3.513 & 177 & 841650 & 1.681 & 13 & 324284 \\
25 & 2.441 & 18 & 301053 & 1.694 & 19 & 339487 \\
26 & 2.415 & 25 & 374594 & 1.711 & 16 & 326141 \\
27 & 2.215 & 7 & 264087 & 2.238 & 5 & 270104 \\
28 & 1.919 & 14 & 272577 & 1.804 & 13 & 273908 \\
29 & 2.228 & 8 & 212822 & 1.936 & 6 & 221706 \\
30 & \Timeout & 18058 & 119444044 & \Timeout & 47925 & 497445022 \\
31 & \Timeout & 23283 & 178289620 & 2.670 & 30 & 532937 \\
32 & 4.870 & 63 & 913320 & \Timeout & 166635 & 858563943 \\
33 & \Timeout & 15181 & 147948821 & \Timeout & 67148 & 763053956 \\
34 & \Timeout & 39057 & 303289935 & 9.809 & 1726 & 14322164 \\
35 & 4.376 & 52 & 904626 & \Timeout & 117958 & 1084029060 \\
36 & 4.396 & 114 & 1628267 & 4.487 & 197 & 1934868 \\
37 & \Timeout & 21286 & 132575730 & 3.146 & 35 & 542375 \\
38 & 4.577 & 79 & 1045936 & 3.162 & 51 & 856676 \\
39 & \Timeout & 9040 & 99188489 & \Timeout & 39272 & 399951117 \\
40 & 2.930 & 1 & 34787 & 2.915 & 1 & 69542 \\
41 & 5.303 & 40 & 790942 & 3.325 & 44 & 834269 \\
42 & \Timeout & 12605 & 148700057 & \Timeout & 42626 & 406830329 \\
43 & 9.819 & 361 & 5325834 & 6.875 & 373 & 5490353 \\
44 & 5.528 & 116 & 1336494 & 4.065 & 37 & 789395 \\
45 & \Timeout & 8415 & 111542169 & \Timeout & 27352 & 334646024 \\
46 & 6.539 & 248 & 2498691 & 4.149 & 80 & 1240475 \\
47 & \Timeout & 22179 & 282707631 & \Timeout & 34120 & 451342876 \\
48 & \Timeout & 19778 & 164139781 & \Timeout & 51875 & 649868285 \\
49 & \Timeout & 15047 & 208250589 & \Timeout & 32003 & 508583392 \\
50 & 0.021 & 0 & 24 & 0.021 & 0 & 29 \\
51 & 0.076 & 0 & 192 & 0.054 & 0 & 209 \\
52 & 0.055 & 0 & 521 & 0.062 & 0 & 525 \\
53 & 0.366 & 6 & 2065 & 0.235 & 6 & 2266 \\
54 & 0.144 & 0 & 1030 & 0.416 & 0 & 988 \\
55 & 0.155 & 2 & 1699 & 0.333 & 2 & 1766 \\
56 & 0.212 & 0 & 2499 & 0.284 & 0 & 2477 \\
57 & 0.279 & 0 & 4201 & 0.226 & 0 & 4440 \\
58 & 0.359 & 0 & 3124 & 0.303 & 0 & 3259 \\
59 & 0.248 & 0 & 4518 & 0.257 & 0 & 4559 \\
60 & 0.246 & 2 & 4170 & 0.301 & 0 & 3489 \\
61 & 0.229 & 0 & 7555 & 0.223 & 1 & 8870 \\
62 & 0.303 & 2 & 12954 & 0.416 & 3 & 13698 \\
63 & 0.378 & 1 & 10738 & 0.603 & 1 & 10691 \\
64 & 0.459 & 2 & 28017 & 0.638 & 12 & 51083 \\
65 & 0.721 & 7 & 57191 & 0.888 & 6 & 54937 \\
66 & 1.134 & 12 & 127865 & 1.490 & 6 & 107939 \\
67 & 1.014 & 10 & 124827 & 0.877 & 13 & 117966 \\
68 & 1.499 & 8 & 111993 & 1.572 & 6 & 114614 \\
69 & 0.351 & 0 & 6613 & 0.310 & 0 & 6631 \\
70 & 0.693 & 20 & 82813 & 0.882 & 16 & 87958 \\
71 & 0.588 & 12 & 92658 & 0.932 & 12 & 98391 \\
 % let your experiment script write directly
                            % into this file, making sure every number
                            % in a column has the _same_ number of decimals
  \end{tabular}
  \caption{}
  \label{tab:res:sat}
\end{table}


\subsection{Discussion}
\label{evaluation:discussion}



\section{Conclusions and Future Work}
\label{conclusions}

\bibliographystyle{abbrv}
\bibliography{astra}

\appendix
\section{Source Code}
\label{sec:source-code}


This appendix presents the source code for the implementation described in \Chapref{sec:implementation}.





\end{document}

%%% Local Variables:
%%% mode: latex
%%% TeX-master: t
%%% End:

% OscaR source code:
% https://bitbucket.org/oscarlib/oscar/src/40e25aafba8f9b0ab06029449350a2a9d1614854/oscar-algo/src/main/scala/oscar/algo/reversible/ReversibleSparseBitSet.scala?at=dev&fileviewer=file-view-default
% https://bitbucket.org/oscarlib/oscar/src/40e25aafba8f9b0ab06029449350a2a9d1614854/oscar-cp/src/main/scala/oscar/cp/constraints/tables/TableCT.scala?at=dev&fileviewer=file-view-default3

% course note in constraint programming
% http://user.it.uu.se/~pierref/courses/COCP/slides/CourseNotes.pdf

% M-x reftex-parse-all
% F1 b
% M-x customize-group reftex

% Hash Functions
% https://en.wikipedia.org/wiki/Pairing_function
% https://www.cs.hmc.edu/~geoff/classes/hmc.cs070.200101/homework10/hashfuncs.html
% http://stackoverflow.com/questions/37918951/what-is-a-minimal-hash-function-for-a-pair-of-ints-that-has-low-chance-of-collis