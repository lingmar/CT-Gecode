\documentclass[a4paper,11pt]{article}
\usepackage{fullpage}
\usepackage{astra}
%\usepackage{etoolbox}\AtBeginEnvironment{algorithmic}{\small‌​}
%\usepackage{algorithm2e}
%\usepackage{algorithmicx}
%\input{macros}

% Silly but saves space
\newcommand{\T}[1]{\texttt{#1}}

\newcommand{\Timeout}{600.00} % CPU seconds
\newcommand{\Todo}[1]{{\color{blue}Todo: #1}}
\newcommand{\Secref}[1]{Section~\ref{#1}}
\newcommand{\Chapref}[1]{Section~\ref{#1}}
\newcommand{\Table}{\Constraint{Table}~}
\newcommand{\Extensional}{\Constraint{Extensional}~}

\newcommand{\Method}[2]{\textbf{method~}\mathrm{{#1}}({#2})}
\newcommand{\MethodReturn}[3]{\textbf{method~}\mathrm{{#1}}({#2})\textbf{\ : \ {#3}}}
\newcommand{\Class}{\textbf{Class~}}
\newcommand{\Constructor}{\textbf{constructor~}}

\newcommand{\Dom}[1]{\text{dom}({#1})}

% SparseBitSet
\newcommand{\Words}{\texttt{words}}
\newcommand{\Index}{\texttt{index}}
\newcommand{\Mask}{\texttt{mask}}
\newcommand{\Limit}{\texttt{limit}}
\newcommand{\SparseBitSet}{\texttt{SparseBitSet}}
\newcommand{\Offset}{\texttt{offset}}

% CT Propagator
\newcommand{\Scp}{\texttt{scp}}
\newcommand{\CurrTable}{\texttt{currTable}}
\newcommand{\Sval}{\texttt{S^{val}}}
\newcommand{\Ssup}{\texttt{S^{sup}}}
\newcommand{\LastSizes}{\texttt{lastSize}}
\newcommand{\Supports}{\texttt{supports}}
\newcommand{\Residues}{\texttt{residues}}

% Pseduo code
\newcommand{\ForEach}[1]{\textbf{foreach } {#1} \textbf{ do }}
\newcommand{\ForEachTo}[3]{\textbf{foreach } {#1} \textbf{ from } {#2} 
  \textbf{ to } {#3} \textbf{ do }}
\newcommand{\ForEachDownTo}[3]{\textbf{foreach } {#1} \textbf{ from } {#2} 
  \textbf{ downto } {#3} \textbf{ do }}
\newcommand{\Break}{\textbf{break~}}
\newcommand{\While}[1]{\textbf{while~} {#1} \textbf{~do~}}

\renewcommand{\algorithmicforall}{\textbf{Function}}
\renewcommand{\algorithmicdo}{}
\renewcommand{\algorithmicwhile}{\textbf{foreach}}

\newcommand{\Func}[2]{\FORALL{#1(#2)}}
\newcommand{\FuncRet}[3]{\FORALL{#1(#2) \ : \ \textbf{#3}}}
\newcommand{\Endfunc}{\ENDFOR}
\newcommand{\To}{~\bf{to}~}
\newcommand{\Downto}{~{\bf{downto}}~}
\newcommand{\For}[3]{\FOR{${#1} \leftarrow {#2} \To {#3}$ \textbf{do}}}
\newcommand{\ForDown}[3]{\FOR{${#1} \leftarrow {#2} \Downto {#3}$ \textbf{do}}}
\newcommand{\FOREACH}[1]{\WHILE{{#1} \textbf{do}}}
\newcommand{\ENDFOREACH}{\ENDWHILE}


\newlength\myindent
\setlength\myindent{2em}
\newcommand\bindent{%
  \begingroup
  \setlength{\itemindent}{\myindent}
  \addtolength{\algorithmicindent}{\myindent}
}
\newcommand\eindent{\endgroup}

\newcommand{\INDSTATE}[1][1]{\STATE\hspace{#1\algorithmicindent}}
\newcommand{\INDRETURN}[1][1]{\STATE\hspace{#1\algorithmicindent}\textbf{return~}}
\newcommand{\INDIF}[2][1]{\STATE\hspace{#1\algorithmicindent}
  \textbf{if~}{#2}\textbf{~then}}
\newcommand{\INDELSE}[1][1]{\STATE\hspace{#1\algorithmicindent}\textbf{else~}}
\newcommand{\INDELSEIF}[2][1]{\STATE\hspace{#1\algorithmicindent}
  \textbf{else if~}{#2}\textbf{~then}}



\title{\textbf{Implementation and Evaluation of a\\
    Compact Table Propagator in Gecode
  }
}

\author{Linnea Ingmar} % replace by your name(s)

%\date{Month Day, Year}
\date{\today}

\begin{document}

\maketitle

\tableofcontents

\newpage

\section{Introduction}
\label{intro}

In Constraint Programming (CP), every constraint is associated with a propagator
algorithm. The propagator algorithm filters out impossible values for the variables
related to the constraint. For the \Table constraint, several propagator
algorithms are known. In 2016, a new propagator algorithm for the \Table
constraint was published~\cite{DBLP:conf/cp/DemeulenaereHLP16}, called Compact Table (CT).
Preliminary results indicate that CT outperforms the previously known algorithms.
There has been no attempt to implement CT in the constraint solver Gecode~\cite{Gecode}, and consequently its performance in Gecode is unknown.

\subsection{Goal}
\label{intro:goal}
The goal of this thesis is to implement a CT progatator
algorithm for the \Table constraint in Gecode,
and to evaluate its performance with respect to the existing propagators.

\subsection{Contributions}
\label{intro:contributions}

\Todo{State the contributions, perhaps as a bulleted list, referring to the different
parts of the paper, as opposed to giving a traditional outline. (As suggested
by Olle Gallmo.)}

This thesis contributes with the following:

\begin{itemize}
  \item The relevant preliminaries have been covered in \Chapref{bg}.
  \item The algorithms presented in~\cite{DBLP:conf/cp/DemeulenaereHLP16} have been modified to suit the
    target constraint solver Gecode, and are presented and explained in 
    \Chapref{algorithms}.
  \item The CT algorithm has been implemented in Gecode, see \Chapref{implementation}.
  \item The performance of the CT algorithm has been evaluated, see \Chapref{evaluation}.
  \item ...
\end{itemize}

\section{Background}
\label{bg}

% Definiera alla begrepp som används senare

This chapter provides a background that is relevant for the
following chapters. It is divided into three parts: \Secref{bg:cp}
introduces Constraint Programming. \Secref{bg:gecode} gives an overview
of Gecode, a constraint solver. Finally, \Secref{bg:table} introduces
the \Table constraint.

\subsection{Constraint Programming}
\label{bg:cp}
This section introduces the concept of Constraint Programming (CP).

CP is a programming paradigm that is used for solving
combinatorial problems. A problem is
modelled as a set of \emph{constraints} and a
set of \emph{variables} with possible values. The possible values of 
a variable is called the \emph{domain} of the variable.
All the variables are to be assigned a value
from their domains, so that all the constraints of the problem
are satisfied. There need not
exist a solution to a given problem, and a solution need not
be unique; zero or more assignments of values to the variables may 
satisfy the constraints.
One small example is the set of variables~$\Set{x,y}$ that both have
the domain~$\Set{1,2,3}$,
together with the constraint~$x<y$. This problem has several solutions, 
namely~$(x,y) = (1,2), (x,y) = (1,3)$ and~$(x,y) = (2,3)$.
If, on the other hand, the domains of~$x$ and~$y$ would have been~$\Set{2,3}$
and~$\Set{1,2}$, respectively, then there would exist no solution since none of
the possible assignments of~$x$ and~$y$ would have satisfied the constraint~$x<y$.

Sometimes the solution should not only satisfy the set of constraints for the
problem, but should also be optimal in some way; it should either maximise or
minimise some given function. Take the same small example as above, with
the domains of~$x,y$ being~$\Set{1,2,3}$, and suppose that the solution
should maximise the sum of~$x$ and~$y$. Then the solution is~$(x,y) = (2,3)$.

A constraint solver is a software that solves constraint problems.
The solving of a problem consists of generating a search tree by branching
on possible values for the variables. At each node in the search tree,
the solver removes impossible values from the domains of the variables,
that is, values that the solver detects
would violate at least one of the constraints if the
variables were to be assigned to any of those values. This filtering
process is called \emph{propagation}. Each constraint that the solver supports
is associated with at least one propagator algorithm, whose task
is to detect impossible values for the constraint. %% This thesis...
A solution is found when all the variables have been assigned a value
that 

% Intuition
% Example
% Definitions
% - CSP
% - 
% Propagation
% - Constistency levels (value-, bounds-, domain-)
% 


\subsection{Gecode}
\label{bg:gecode}
Gecode~\cite{Gecode} is a popular constraint programming solver written in C++.

% definiera de delar av Gecodes API som dyker upp senare, såsom propagate(), status messages

\subsection{The \Table Constraint}
\label{bg:table}
The \Table constraint, called \Extensional in Gecode,
explicitly expresses the possible combinations of values for the variables as a
sequence of $n$-tuples.

\subsection{Compact-Table Propagator}
\label{bg:ct}
% Beskriv huvudidéerna
% Komplexitet? O(r*d*t) per table constraint along a branch in the search tree

\subsection{Sparse Bit-Set}
\label{bg:sbc}
% Beskriv idén

\section{Algorithms}
\label{algorithms}

% Section 3 bör beskriva din design i detalj men samtidigt inte på C++-nivå. Jag gillar att se sjok av pseudokod inbäddade i text som förklarar pseudokoden. Man kan skriva text mellan sjoken och/eller i caption till algorithm-omgivningen. Något som jag också gillar är stepwise refinement, dvs. att först visa en enkel men korrekt version, och sedan en eller flera mer sofistikerade, optimerade versioner. Den pseudokod som du har skrivit passar bra i Section 3, men bryt gärna upp åtminstone Class CT-Propagator i flera stycken algorithm-omgivningar.

This chapter presents the algorithms that are used in the implementation of the
CT propagator in \Chapref{implementation}.

\subsection{Sparse Bit-Set}
This section describes the class~$\SparseBitSet$ which is the main datastructure
in the CT algorithm for maintaining the supports.

\begin{algorithm}[h]
  \begin{algorithmic}[1]  % comment [1] away to drop the line numbers
    \STATE $\Class$ SparseBitSet
    \item[]
      \STATE $\Words$: array of long \COMMENT{$\tt{\Words.length = p}$}
      \STATE $\Index$: array of int \COMMENT{$\tt{\Index.length = p}$}
      \STATE $\Limit$: int 
      \STATE $\Mask$: array of long \COMMENT{$\tt{\Mask.length = p}$}
    \item[]
      \FuncRet{isEmpty}{}{Boolean}
      \RETURN{$\Limit = -1$}
      \Endfunc
    \item[]
      \Func{clearMask}{}
      \For{i}{0}{\Limit}
        \STATE $\texttt{offset} \leftarrow \Index[i]$
        \STATE $\texttt{mask}[\texttt{offset}] \leftarrow 0^{64}$
      \ENDFOR
      \Endfunc

    % \item[] 
    %   \Func{reverseMask}{}  \COMMENT{Not currently used in CT algorithm}
    %   \STATE $\ForEachTo{i}{0}{\Limit}$
    %   \STATE $\texttt{offset} \leftarrow \Index[i]$
    %   \STATE $\texttt{mask}[\texttt{offset}] \leftarrow 
    %   {\raise.17ex\hbox{$\scriptstyle\mathtt{\sim}$}}
    %   \texttt{mask}[\texttt{offset}]$ \COMMENT{bitwise NOT}
      %\Endfunc

    \item[]
      \Func{addToMask}{m: array of long}
      \For{i}{0}{\Limit}
      \STATE $\texttt{offset} \leftarrow \Index[i]$
      \STATE $\texttt{mask}[\texttt{offset}] \leftarrow \texttt{mask}[\texttt{offset}] \ | \ 
      \texttt{m}[\texttt{offset}]$ \COMMENT{bitwise OR}
      \ENDFOR
      \Endfunc

    \item[]
      \Func{intersectWithMask}{}
      %\FOR{$i \leftarrow \Limit \Downto 0$}
      \ForDown{i}{\Limit}{0}
      \STATE $\texttt{offset} \leftarrow \Index[i]$
      \STATE $w \leftarrow \Words[\texttt{offset}] \ \& \ \Mask[\texttt{offset}]$ \COMMENT{bitwise AND}
      \IF{$w \neq \Words[\texttt{offset}]$}
      \STATE $\Words[\texttt{offset}] \leftarrow w$
      \IF{$w = 0^{64}$}
      \STATE $\Index[i] \leftarrow \Index[\Limit]$
      %\STATE $\Index[\Limit] \leftarrow \texttt{offset}$
      \STATE $\Limit \leftarrow \Limit - 1$
      \ENDIF
      \ENDIF
      \ENDFOR
      \Endfunc

    \item[]
      \FuncRet{intersectIndex}{m: array of long}{int}
      \For{i}{0}{\Limit}%{$i \leftarrow 0 \To \Limit$}
      \STATE $\Offset \leftarrow \Index[i]$
      \IF{$\Words[\Offset] \ \& \ m[\Offset] \neq 0^{64}$}
      \RETURN{$\Offset$}
      \ENDIF
      \ENDFOR
      \RETURN{$-1$}
      \Endfunc
            
    \end{algorithmic}
  \caption{Pseudo code for the class SparseBitSet.}
  \label{algo:sparse}
\end{algorithm}

\subsection{Compact-Table (CT) Algorithm}
This section describes the CT algorithm.

% \begin{algorithm}[h]
%   \begin{algorithmic}[1]  % comment [1] away to drop the line numbers
%     \STATE $\Class$ CT-Propagator
%     \item[]
%       \STATE $\Scp$: array of variables
%       \STATE $\CurrTable$: SparseBitSet \COMMENT{Current supported tuples}
%       \STATE $\Supports$   \COMMENT{$\Supports[x,a]$ is the bit-set of supports for $(x,a)$}

%       \STATE $\LastSizes$: array of int \COMMENT{$\LastSizes[x]$ is the last size of the domain of $x$.c}

%       \STATE $\Residues$  \COMMENT{$\Residues[x,a]$ is the last found support for $(x,a)$. 
%       No residues yet!}
            
%     \item[]
%       \Func{initialise}{\texttt{variables}, \texttt{tuples}} %\COMMENT{Initialise the propagator}
%       \STATE $\Scp \leftarrow \texttt{variables}$
%       \FOREACH{$x \in \Scp$} %\COMMENT{Initial bounds propagation}
%       \STATE $\Dom{x} \leftarrow \Dom{x} \setminus  
%       \Set{a \in \Dom{x} : a > \texttt{tuples}.max()}$
%       \STATE $\Dom{x} \leftarrow \Dom{x} \setminus 
%       \Set{a \in \Dom{x} : a < \texttt{tuples}.min()}$
%       %\STATE[2] $\LastSizes[x] \leftarrow |\Dom{x}|$
%       \ENDFOREACH

%       \STATE $\texttt{size} \leftarrow sum\Set{|\Dom{x}| : x \in \Scp}$
%       \STATE $\texttt{ntuples} \leftarrow \texttt{tuples}.size()$ \COMMENT{Number of tuples}
%       \STATE $\Supports \leftarrow$ BitSets$(\texttt{size}, \texttt{ntuples})$
%       \COMMENT{bit-set matrix with \texttt{size} rows and \texttt{ntuples} columns}
%       \STATE $\T{no\_supports} \leftarrow 0$
      
%       \FOREACH{$t \in \T{tuples}$}
%       \STATE $\T{supported} \leftarrow \T{true}$
%       \FOREACH{$x \in \Scp$}
%       \IF{$t[x] \notin \Dom{x}$}
%       \STATE $\T{supported} \leftarrow \T{false}$
%       \STATE $\textbf{break}$ \COMMENT{Exit loop}
%       \ENDIF
%       \ENDFOREACH
%       \IF{$\T{supported}$} 
%         \STATE $\T{no\_supports} \leftarrow \T{no\_supports} + 1$
%         \STATE $\Supports.$setElemsInColumn($\T{no\_supports}$, $t$) 
%       \ENDIF
%       \ENDFOREACH
%       \COMMENT{Mark tuple as supported}

%       \STATE $\Supports.\text{trimToWidth}(\T{no\_supports})$ \COMMENT{Keep only the first $\T{no\_supports}$ bits for each row}
%       \STATE $\CurrTable \leftarrow \SparseBitSet(\T{no\_supports})$ 
%       \COMMENT{$\SparseBitSet$ with \T{no\_supports} bits}
%       \Endfunc

%     \item[]
%       \Func{updateTable}{}   
%       \FOREACH{$x \in \Scp \text{~such that~} |\Dom{x}| \neq \LastSizes[x]$}
%         \STATE $\LastSizes[x] \leftarrow |\Dom{x}|$
%         \STATE $\CurrTable$.clearMask() 
%         \FOREACH{$a \in \Dom{x}$} %\COMMENT{No incremental update yet!}
%           \STATE $\CurrTable$.addToMask($\Supports[x,a]$)
%           \STATE $\CurrTable$.intersectWithMask()
%           \IF{$\CurrTable$.isEmpty()}
%             \STATE $\Break$
%           \ENDIF
%         \ENDFOREACH
%      \ENDFOREACH

%     \item[]
%       \STATE $\Method{filterDomains}{}$
%       \STATE $\T{no\_assigned} \leftarrow 0$
%       \STATE $\ForEach{x \in \Scp}$
%       \STATE[2] $\ForEach{a \in \Dom{x}}$
%       \STATE[3] $\texttt{index} \leftarrow \CurrTable$.intersectIndex($\Supports[x,a]$)
%       \INDIF[3]{$\texttt{index~} \neq -1$}
%       \STATE[4] \todo{save index in residues} \COMMENT{No residues yet!}
%       \INDELSE[3]
%       \STATE[4] $\Dom{x} \leftarrow \Dom{x} \setminus \Set{a}$
%       \INDIF[4]{$|\Dom{x}| = 0$}
%       \INDRETURN[5]{$Failed$}
%       \INDELSEIF[4]{$|\Dom{x}| > 1$}
%       \STATE[5] $\T{no\_assigned} \leftarrow no\_assigned + 1$
%       \INDIF{$\T{no\_assinged} \leq 1$}
%       \INDRETURN[2]{$Subsumed$}
%       \INDELSE{}
%       \INDRETURN[2]{$Fixpoint$}

%     % \item[]
%     %   \STATE $\Method{propagate}{}$
%     %   \STATE updateTable()
%     %   \INDIF{$\CurrTable$.isEmpty()}
%     %   \INDRETURN[2]{$Failed$}
%     %   \INDRETURN{filterDomains()}

%     \end{algorithmic}
%   \caption{Pseudo code for CT propagator class.}
%   \label{algo:CT}
% \end{algorithm}

\subsubsection{Propagator Status Messages}
A propagtor signals a status message after propagation.
For CT there are three possible status messages;~\emph{Fail},
\emph{Subsumed} and~\emph{Fixpoint}.

\paragraph{Fail.}
A propagator must correctly signal failure
if it has decided that the constraint is unsatisfiable for the input store~$S$. At the latest a propagator must be
able to decide whether or not a~$S$ is a solution store when all
variables have been assigned.
CT has two ways of detecting a failure: either when all bits in~$\CurrTable$
are set to zero -- meaning that none of the tuples are valid, or when the
size of the domain of a variable is zero.

\paragraph{Subsumption.}
A propagator is not allowed to signal subsumption if it could propagate further
at a later point. At the latest, a propagator must signal subsumption if all the variables
are assigned in~$S$ and a~$S$ is a solution store.
CT signals subsumption when at most one variable is not assigned,
since this is the pointe where no more propagation can be made.

\paragraph{Fixpoint.}
A propagator is not allowed to claim that it has computed a fixpoint if
it could still propagate.
CT always computes a fixpoint if it is not subsumed or wipes out the domain of a variable.
To understand this, consider two consequtive calls to propagate().
Let~$T$ be the set of valid tuples and~$x$ be the set of variables.
The first time propagate() is executed, a (possibly empty) subset~$T_r$ of~$T$
is invalidated (that is, the corresponding bits in $\CurrTable$ are set to $0$)
in updateTable().
Assuming $\CurrTable$ is not empty after the return of updateTable(), which would
cause a backtrack in the search, filterDomains() is called.
This method removes a (possibly empty) set of values~$V_i$ from the domain 
of each variable~$x_i$.
Each value $v_i \in V_i$ has the property that a subset~$T_{v_i}$ 
of the tuples in~$T_r$ are the last supports for~$(x_i,v_i)$. In other words,
$T \setminus T_{v_i}$ does not contain a support for~$(x_i,v_i)$.
So in the second call to updateTable(), no tuples are invalidated,
because none of the tuples in~$T \setminus T_r$ is a support for any variable-value
pair~$(x_i, v_i) \in \Set{x_i} \times V_i, i \in \Set{1...|x|}$.
Hence, the second call to propagate() does not give
any further propagation.

\section{Implementation}
\label{implementation}

% Section 4 blir nog mindre intressant än Section 3 och 5, men där kan du skriva om sånt som är specifikt för C++ och Gecode för att algoritmerna i Section 3 ska fungera, precis som du har börjat göra. Det är också en bra plats för detaljer som sopats under mattan i pseudokoden, t.ex. exakt hur du mappar (x,a) till rätt element i supports och residues, med hashtabell eller så.

This chapter describes an implementation of the CT propagator using the algorithms
presented in \Chapref{algorithms}. The implementation was made in the C++ programming
language in the Gecode library.

The bit-set matrix $\Supports$ is static and can be shared between all solution spaces.

The bit-set $\CurrTable$ changes dynamically during propagation and must therefore be copied for
every new space. Can save memory by only copying the non-zero words.

No need to save the $\T{tuples}$ as a field in the propagator class as all
the necessary information is encoded in $\CurrTable$ and $\Supports$.


% Can't modify the value of the variable while iterating over it
% when using an iterator for a view, the view cannot be modified (or, in C++ lingua: modifying the variable invalidates the iterator).

% The motivation to iterate over range sequences rather than individual values is efficiency:
% as there are typically less ranges than indvidual values, iteration over ranges can be more efficient.

% Sharing of domain and iterators (argument false in inter_v)

%http://www.gecode.org/doc/4.4.0/reference/classGecode_1_1Iter_1_1Values_1_1BitSet.html

% First perform bounds propagation

% Staging p. 324

% Region (memory allocation)

% Multimap for hashing rows?

\section{Evaluation}
\label{evaluation}
This chapter presents the evaluation of the implementation of the CT propagator
presented in \Chapref{implementation}. In \Secref{evaluation:setup},
the evaluation setup is described. In \Secref{evaluation:results} presents
the results of the evaluation. The results are discussed in \Secref{evaluation:discussion}.

\subsection{Evaluation Setup}
\label{evaluation:setup}
\subsection{Results}
\label{evaluation:results}
\subsection{Discussion}
\label{evaluation:discussion}

\section{Conclusions and Future Work}
\label{conclusions}

\bibliographystyle{abbrv}
\bibliography{astra}

\appendix
\section{Source Code}
\label{source code}

This appendix presents the source code for the implementation described in \Chapref{implementation}.

\end{document}

%%% Local Variables:
%%% mode: latex
%%% TeX-master: t
%%% End:
