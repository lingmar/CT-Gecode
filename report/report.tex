\documentclass[a4paper,11pt]{article}
\usepackage{fullpage}
\usepackage{astra}

%\input{macros}

\newcommand{\Timeout}{600.00} % CPU seconds

\title{\textbf{Implementation and Evaluation of a\\
    Compact Table Propagator in Gecode
  }
}

\author{Linnea Ingmar} % replace by your name(s)

%\date{Month Day, Year}
\date{\today}

\begin{document}

\maketitle

\tableofcontents

\newpage

\section{Introduction}
\label{intro}
The goal of this thesis is to implement a Compact Table (CT) progatator
algorithm for the \Constraint{Table} constraint in Gecode,
an open-source constraint solver, and
to evaluate its performance with respect to the existing propagators.

\section{Background}
\label{bg}
This chapter gives an overview of preliminaries that are relevant for the
following chapters. It is divided into three parts: Section~\ref{bg:cp}
introduces Constraint Programming. Section~\ref{bg:gecode} gives an overview
of Gecode, a constraint solver. Finally, Section~\ref{bg:table} introduces
the \Constraint{Table} constraint.

\subsection{Constraint Programming}
\label{bg:cp}
This section introduces the concept of Constraint Programming (CP).
\subsection{Gecode}
\label{bg:gecode}
Gecode~\cite{Gecode} is a popular constraint programming solver written in the
C++ programming language. 
\subsection{The \Constraint{Table} Constraint}
\label{bg:table}
The \Constraint{Table} constraint?

\section{Algorithms}
\label{algorithms}

\section{Implementation}
\label{implementation}

\section{Evaluation}
\label{evaluation}
\subsection{Evaluation Setup}
\label{evaluation:setup}
\subsection{Results}
\label{evaluation:results}
\subsection{Discussion}
\label{evaluation:discussion}

\section{Conclusions and Future Work}
\label{conclusions}

\bibliographystyle{abbrv}
\bibliography{astra}

\appendix
\section{Source Code}
\label{source code}

\end{document}

%%% Local Variables:
%%% mode: latex
%%% TeX-master: t
%%% End:
