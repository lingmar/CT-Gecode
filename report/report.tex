\documentclass[a4paper,11pt]{article}
\usepackage{fullpage}
\usepackage{astra}

%\input{macros}

\newcommand{\Timeout}{600.00} % CPU seconds
\newcommand{\Todo}[1]{{\color{blue}Todo: #1}}
\newcommand{\Secref}[1]{Section~\ref{#1}}
\newcommand{\Chapref}[1]{Chapter~\ref{#1}}
\newcommand{\Table}{\Constraint{Table}}
\newcommand{\Extensional}{\Constraint{Extensional}}

\newcommand{\Method}[2]{\textbf{method~}\text{{#1}}({#2})}
\newcommand{\MethodReturn}[3]{\textbf{method~}\text{{#1}}({#2})\textbf{\ : \ {#3}}}
\newcommand{\Class}{\textbf{Class~}}
\newcommand{\Constructor}{\textbf{constructor~}}

% SparseBitSet
\newcommand{\Words}{\texttt{words}}
\newcommand{\Index}{\texttt{index}}
\newcommand{\Mask}{\texttt{mask}}
\newcommand{\Limit}{\texttt{limit}}
\newcommand{\SparseBitSet}{\texttt{SparseBitSet}}

% CT Propagator
\newcommand{\Scp}{\texttt{scp}}
\newcommand{\CurrTable}{\texttt{currTable}}
\newcommand{\Sval}{\texttt{S^{val}}}
\newcommand{\Ssup}{\texttt{S^{sup}}}
\newcommand{\LastSizes}{\texttt{lastSizes}}
\newcommand{\Supports}{\texttt{supports}}
\newcommand{\Residues}{\texttt{residues}}


\title{\textbf{Implementation and Evaluation of a\\
    Compact Table Propagator in Gecode
  }
}

\author{Linnea Ingmar} % replace by your name(s)

%\date{Month Day, Year}
\date{\today}

\begin{document}

\maketitle

\tableofcontents

\newpage

\section{Introduction}
\label{intro}

In Constraint Programming (CP), every constraint is associated with a propagator
algorithm. The propagator algorithm filters out impossible values for the variables
related to the constraint. For the \Table constraint, several propagator
algorithms are known. In 2016, a new propagator algorithm for the \Table
constraint was published~\cite{DBLP:conf/cp/DemeulenaereHLP16}, called Compact Table (CT).
Preliminary results indicate that CT outperforms the previously known algorithms.
There has been no attempt to implement CT in the constraint solver Gecode~\cite{Gecode}, and consequently its performance in Gecode is unknown.

\subsection{Goal}
\label{intro:goal}
The goal of this thesis is to implement a CT progatator
algorithm for the \Table constraint in Gecode,
and to evaluate its performance with respect to the existing propagators.

\subsection{Contributions}
\label{intro:contributions}

\Todo{State the contributions, perhaps as a bulleted list, referring to the different
parts of the paper, as opposed to giving a traditional outline. (As suggested
by Olle Gallmo.)}

This thesis contributes with the following:

\begin{itemize}
  \item The relevant preliminaries have been covered in \Chapref{bg}.
  \item The algorithms presented in~\cite{DBLP:conf/cp/DemeulenaereHLP16} have been modified to suit the
    target constraint solver Gecode, and are presented and explained in 
    \Chapref{algorithms}.
  \item The CT algorithm has been implemented in Gecode, see \Chapref{implementation}.
  \item The performance of the CT algorithm has been evaluated, see \Chapref{evaluation}.
  \item ...
\end{itemize}

\section{Background}
\label{bg}

This chapter provides a background that is relevant for the
following chapters. It is divided into three parts: \Secref{bg:cp}
introduces Constraint Programming. \Secref{bg:gecode} gives an overview
of Gecode, a constraint solver. Finally, \Secref{bg:table} introduces
the \Table constraint.

\subsection{Constraint Programming}
\label{bg:cp}
This section introduces the concept of Constraint Programming (CP).
\subsection{Gecode}
\label{bg:gecode}
Gecode~\cite{Gecode} is a popular constraint programming solver written in C++.
\subsection{The \Table Constraint}
\label{bg:table}
The \Table constraint, called \Extensional in Gecode,
explicitly expresses the possible combinations of values for the variables as a
sequence of $n$-tuples.

\section{Algorithms}
\label{algorithms}
This chapter presents the algorithms that are used in the implementation of the
CT propagator in \Chapref{implementation}.

\subsection{Sparse Bit-Set}
This section describes the class~$\SparseBitSet$ which is the main datastructure
in the CT algorithm for maintaining the supports.

\begin{algorithm}[h]
  \begin{algorithmic}[1]  % comment [1] away to drop the line numbers
    \STATE $\Class$ SparseBitSet
    \item[]
      \STATE $\Words$: array of long \COMMENT{$\tt{\Words.length = p}$}
      \STATE $\Index$: array of int \COMMENT{$\tt{\Index.length = p}$}
      \STATE $\Limit$: int 
      \STATE $\Mask$: array of long \COMMENT{$\tt{\Mask.length = p}$}
    \item[]
      \STATE \MethodReturn{isEmpty}{}{Boolean}
      \RETURN $\Limit = -1$
    \item[]
      \STATE \Method{reverseMask}{} 
      \STATE TBA
      %\FORALL{i} \STATE{hej} \ENDFOR
    \item[]
      \STATE \Method{addToMask}{m: array of long}
      \STATE TBA
    \item[]
      \STATE \Method{intersectWithMask}{}
      \STATE TBA
    \item[]
      \STATE \Method{intersectWithMask}{}
      \STATE TBA
    \item[]
      \STATE \MethodReturn{intersectIndex}{m: array of long}{int}
      \STATE TBA
            
    \end{algorithmic}
  \caption{Pseudo code for the class SparseBitSet.}
  \label{algo:sparse}
\end{algorithm}

\subsection{Compact-Table (CT) Algorithm}
This section describes the CT algorithm.

\begin{algorithm}[h]
  \begin{algorithmic}[1]  % comment [1] away to drop the line numbers
    \STATE $\Class$ CT-Propagator
    \item[]
      \STATE $\Scp$: array of variables
      \STATE $\CurrTable$: SparseBitSet \COMMENT{Current supported tuples}
      \STATE $\LastSizes$: array of int \COMMENT{$\LastSizes[x]$ is the last size of the domain of $x$}
      \STATE $\Supports$: BitSet \COMMENT{$\Supports[x,a]$ is the bit-set of supports for $(x,a)$}
      \STATE $\Residues$  \COMMENT{$\Residues[x,a]$ is the last found support for $(x,a)$}

    \item[]
      \STATE \Method{initialise}{$\tt{variables}$, $\tt{tuples}$}
      \STATE TBA \COMMENT{Initialise global variables}

    \item[]
      \STATE \Method{updateTable}{}
      \STATE TBA \COMMENT{Update $\CurrTable$}
    \item[]
      \STATE \Method{filterDomains}{} 
      \STATE TBA \COMMENT{Filter invalidated values from domains}
    \item[]
      \STATE \Method{propagate}{}
      \STATE TBA \COMMENT{Perform propagation}
    \end{algorithmic}
  \caption{Pseudo code for the class SparseBitSet.}
  \label{algo:sparse}
\end{algorithm}

\section{Implementation}
\label{implementation}
This chapter presents an implementation of the CT propagator using the algorithms
presented in \Chapref{algorithms}.

\section{Evaluation}
\label{evaluation}
This chapter presents the evaluation of the implementation of the CT propagator
presented in \Chapref{implementation}. In \Secref{evaluation:setup},
the evaluation setup is described. In \Secref{evaluation:results} presents
the results of the evaluation. The results are discussed in \Secref{evaluation:discussion}.

\subsection{Evaluation Setup}
\label{evaluation:setup}
\subsection{Results}
\label{evaluation:results}
\subsection{Discussion}
\label{evaluation:discussion}

\section{Conclusions and Future Work}
\label{conclusions}

\bibliographystyle{abbrv}
\bibliography{astra}

\appendix
\section{Source Code}
\label{source code}

This appendix presents the source code for the implementation described in \Chapref{implementation}.

\end{document}

%%% Local Variables:
%%% mode: latex
%%% TeX-master: t
%%% End:
